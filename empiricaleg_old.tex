\documentclass[11pt]{article}

\textwidth 16.5cm
\evensidemargin 0mm
\oddsidemargin0mm
\topmargin-18mm
\textheight 23.0cm

%----------------------------------------------------------------------------------------------------------
% PACKAGES
%----------------------------------------------------------------------------------------------------------
%\usepackage{epsfig}
%\usepackage{draftwatermark}
\usepackage{everypage}
\usepackage{lscape}
\usepackage{amsfonts}
\usepackage{amssymb}
\usepackage{amsmath}
\usepackage{setspace}
\usepackage[stable]{footmisc}
\doublespacing
%\usepackage[bottom]{footmisc}
\usepackage{float}
\usepackage{longtable}
\usepackage[center,bf]{caption}
\usepackage[debug,citecolor=blue]{hyperref}
\usepackage{theorem}
\usepackage[toc,page,title,titletoc,header]{appendix}
\usepackage{tikz}                                             % you only need this when using TikZ graphics
%\usepackage{pdfsync}
\usepackage{graphicx}
\usepackage{float}
\usepackage[final]{pdfpages}
%\usepackage{epsfig}
%\usepackage{3parttable}
%\usepackage{dcolumn}
\DeclareGraphicsExtensions{.png, .pdf, .jpg, .mps, .ps, .pdf, .eps}
\usepackage{pdfpages}
\usepackage{color}
\usepackage{xcolor}
\usepackage{rotating}
\usepackage{multirow}
\usepackage{placeins}

\usepackage{bbm}

\graphicspath{{./figuras/}}

%----------------------------------------------------------------------------------------------------------
% BIBLIOGRAPHY
%----------------------------------------------------------------------------------------------------------
\usepackage{natbib}
%{\bibliographystyle{econometrica}
\renewcommand{\cite}{\citet}

%--------
%%WATER MARK
%\SetWatermarkFontSize{1cm}
%\SetWatermarkScale{2}
%\SetWatermarkLightness{.9}
%\SetWatermarkText{Preliminary Draft}

%--------

%%%%%%%%% NEW COMMAND MACRO %%%%%%%%%%%%%%%%%%%%%%%%%%%%%%

\newcommand{\re}[1]{\smallskip\textsf{\textbf{Respuesta}} \begin{sf}\\
#1 \end{sf} \bigskip}
\newcommand{\pr}[2]{\frac{\partial #1}{\partial #2}}

\newcommand{\notorth}{\ensuremath{\perp\!\!\!\!\!\!\diagup\!\!\!\!\!\!\perp}}%
\newcommand{\orth}{\ensuremath{\perp\!\!\!\perp}}%

%%%%%%%%%%%%%%%%%%%%%%%%%%%%%%%%%%%%%%%%%%%%%%%%%%%%%%%%%%

%\input{ee.sty}

\setcounter{MaxMatrixCols}{10}

\newtheorem{example}{Example}[section]
\newtheorem{assume}{Assumption}
\newtheorem{proposition}{Proposition}[section]
\newtheorem{lemma}{Lemma}[section]
\newtheorem{corollary}{Corollary}[section]
\newtheorem{theorem}{Theorem}[section]
\newtheorem{definition}{Definition}[section]
\setcounter{section}{0}
\makeindex

\newcommand{\notindep}{\ensuremath{\perp\!\!\!\!\!\!\diagup\!\!\!\!\!\!\perp}}
\newcommand{\indep}{\ensuremath{\perp\!\!\!\perp}}
\newcommand{\Perp}{\perp \! \! \! \perp}

\newcommand{\prob}{\mathbb{P}}
\newcommand{\expec}{\mathbb{E}}


\def\inprobHIGH{\,{\buildrel p \over \longrightarrow}\,} 
\def\inprob{\,{\inprobHIGH}\,} 

\def\indistHIGH{\,{\buildrel d \over \longrightarrow}\,} 
\def\indist{\,{\indistHIGH}\,} 

%----------------------------------------------------------------------------------------------------------
% HYPER-REFERENCES CONFIGURATION - PDF
%----------------------------------------------------------------------------------------------------------

\hypersetup{
  colorlinks=true,linkcolor=blue,citecolor=black, filecolor=magenta,
citebordercolor=yellow,
  linkbordercolor = magenta,
  pdftitle={SGL},
  pdfauthor={Rojas},
  pdfcreator={\LaTeX\ with package \flqq hyperref\frqq using Textmate},
  pdfsubject={achievement},
  pdfkeywords={achievement}
}

\usepackage{psfrag}
\usepackage{afterpage}
\usepackage{subfigure}
\usepackage{setspace}
\usepackage{fancyhdr}

\newcommand{\norm}[1]{\left\lVert#1\right\rVert}

\setlength\parindent{0pt}

\begin{document}

\section*{Empirical Example}

\vspace{0.1in}
The empirical example is based on the panel data in Baltagi et al. (2000). The data is suitable to analyze the cigarette demand of 46 American states over 30 years (1963 - 1992). In this section, we consider the model and moment selection problem together. The demand model in Baltagi et al. (2000) is as follows (all prices in real terms): 
\[
\ln C_{it} =  \gamma \ln C_{i,t-1} + \theta \ln P_{it} + \alpha W_{it} + \eta_i +  v_{it}
\] 
where $C_{it}$ is per capita sales of cigarettes, $P_{it}$ is the average retail price of a pack of cigarettes, and $\eta_i$ captures state fixed effect. The vector of all the remaining regressors is denoted as $W_{it}$ above. It includes the minimum price of cigarettes in any neighboring state $Pn_{it}$, per capita disposable income $Y_{it}$, and year dummies. After first-differencing, we obtain 
\[
\Delta \ln C_{it} = \gamma \Delta \ln C_{i,t-1} +  \theta \Delta \ln P_{it} +\alpha \Delta W_{it} +  \Delta v_{it}
\]
 Suppose we are interested in whether to use a dynamic specification of cigarette demand to estimate price elasticity. Then we can project out $\Delta W_{it}$ to rewrite the model as
\[
\widetilde{C}_{it} = \gamma \widetilde{C}_{i,t-1} + \theta \widetilde{P}_{it} + \widetilde{v}_{it} 
\] 
 
 where $\widetilde{C}_{it}$ and $\widetilde{P}_{it}$ are the residuals of $\Delta \ln C_{it}$ and $\Delta \ln P_{it}$ respectively, after projecting out $\Delta W_{it}$.
 
The model selection decision is whether or not to set $\gamma = 0$. The moment selection is whether or not to use $\ln P_{it}$ (after projecting out $\Delta W_{it})$ as an instrument for period $t$.  There are four specifications:  LW, LS, W, and S.\footnote{The specification LW we consider is different from FD-2SLS in Baltagi et al. (2000). Baltagi et al. (2000) consider the following specification without time dummies:
 \[
\Delta \ln C_{it} = \beta_1 \Delta \ln C_{i,t-1} +  \beta_2 \Delta \ln P_{it} +\beta_3 \Delta \ln Pn_{it} +  \beta_4 \Delta \ln Y_{it} + \Delta v_{it}.
\]
Furthermore, they instrument $\Delta \ln C_{i,t-1}$ by the lagged values of exogenous variables $\ln P_{it}, \ln Pn_{it}, \ln Y_{it},$ $\ln P_{i,t-1}, \ln Pn_{i,t-1}, \ln Y_{i,t-1}, \ln P_{i,t-2}, \ln Pn_{i,t-2}, \ln Y_{i,t-2}$. Instead, our specification LW use instrument $\ln C_{i, t-2}$ for $\Delta \ln C_{i,t-1}$ as suggested in Anderson and Hsiao (1982).}
 Both LW and LS include a lagged dependent variable. LW and W designate the letter W for weak exogeneity assumption. Our aim is to use the GFIC to choose between competing 2SLS estimators of $\theta$ on the basis of AMSE. %We can construct all the necessary components as in the previous simulation section.
\newpage


\begin{table}[h!]\centering
 \caption{The Estimates of Price Elasticity (Full Data, 1963-1992, T=30)}
\begin{tabular}{l c c c c }\hline\hline 
 & LW   &      LS   &       W   &      S\\
\hline
$\theta$ & \textcolor{blue}{-0.149} & -0.384 &  -0.067 &  -0.382\\
\hline
GFIC value &0.0115 & 2.3631 & 0.3637 &  2.1788\\
Bias$^2$ & 0.0 & 2.3618 & 0.3509 & 2.1777\\
Variance & 0.0115 & 0.0012 & 0.0128 & 0.0011\\ 
\hline
\hline
\end{tabular}
\end{table}



\begin{table}[h!]\centering
 \caption{The Estimates of Price Elasticity (Shortened data, 1963 - 1974, T=12)}
\begin{tabular}{l c c c c }\hline\hline 
 & LW   &      LS   &       W   &      S\\
\hline
$\theta$ & \textcolor{blue}{-0.313} & -0.518 &  -0.157 &  -0.510\\
\hline
GFIC value &0.0340 & 1.1518 & 0.6553 &  1.2846\\
Bias$^2$ & 0.0 & 1.1495 & 0.6119 & 1.2827\\
Variance & 0.0340 & 0.0023 & 0.0434 & 0.0019\\ 
\hline
\hline
\end{tabular}
\end{table}


\begin{table}[h!]\centering
 \caption{The Estimates of Price Elasticity (Shortened data, 1975 - 1985, T=11)}
\begin{tabular}{l c c c c }\hline\hline 
 & LW   &      LS   &       W   &      S\\
\hline
$\theta$ & \textcolor{blue}{-0.304} & -0.259 &  -0.380 &  -0.280\\
\hline
GFIC value &0.0553 & 2.2200 & 0.2664 &  4.3398\\
Bias$^2$ & 0.0 & 2.2137 & 0.2202 & 4.3348\\
Variance & 0.0553 & 0.0063 & 0.0462 & 0.0050\\ 
\hline
\hline
\end{tabular}
\end{table}

\newpage

\begin{table}[h!]\centering
 \caption{The Estimates of Price Elasticity (Shortened data, 1970 - 1975, T=6)}
\begin{tabular}{l c c c c }\hline\hline 
 & LW   &      LS   &       W   &      S\\
\hline
$\theta$ & -0.341 & \textcolor{blue}{ -0.540} &  0.414 &  -0.448\\
\hline
GFIC value &0.1820 & 0.0106 & 2.3920& 9.2667\\
Bias$^2$ & 0.0 & 0.0 & 2.2985 & 9.2607\\
Variance &0.1820 & 0.0106 & 0.0935 & 0.0060\\ 
\hline
\hline
\end{tabular}
\end{table}

\begin{table}[h!]\centering
 \caption{The Estimates of Price Elasticity (Shortened data, 1968 - 1973, T=6)}
\begin{tabular}{l c c c c }\hline\hline 
 & LW   &      LS   &       W   &      S\\
\hline
$\theta$ & -0.198 & \textcolor{blue}{ -0.545} &  0.244 &  -0.513\\
\hline
GFIC value &0.1564 & 0.0053 & 4.6153& 2.3967\\
Bias$^2$ & 0.0 & 0.0 & 4.4815 & 2.3967\\
Variance &0.1564 & 0.0053 & 0.1338 & 0.0037\\ 
\hline
\hline
\end{tabular}
\end{table}


\begin{table}[h!]\centering
 \caption{The Estimates of Price Elasticity (Shortened data, 1973 - 1978, T=6)}
\begin{tabular}{l c c c c }\hline\hline 
 & LW   &      LS   &       W   &      S\\
\hline
$\theta$ & -0.293 & \textcolor{blue}{ -0.403} &  -0.033 &  -0.425\\
\hline
GFIC value &0.0920 & 0.0163 & 1.6528 & 2.8419\\
Bias$^2$ & 0.0 & 0.0 & 1.5767 & 2.8286\\
Variance &0.0920 & 0.0163 & 0.0761 & 0.0133\\ 
\hline
\hline
\end{tabular}
\end{table}



\end{document}