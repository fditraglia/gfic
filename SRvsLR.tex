\section{Dynamic Panel Example (Short-run Effect v.s. Long-run Effect)}
\label{sec:SRvsLR}

We now evaluate the performance of the GFIC in a simulation experiment based on the dynamic panel example from the preceding section. We are interested in comparing LW (including $y_{it-2}$) and W (excluding $y_{it-2}$) cases. It is a model selection problem of whether $\alpha_2=0$. The simulated covariates and error terms are jointly normal with mean zero and unit variance. Specifically,
	\begin{equation}
		\left[\begin{array}{c}
			x_{i}\\
			\eta_i\\
			v_{i}
	 \end{array} \right]\sim \mbox{iid}\; N\left(\left[\begin{array}{c}0_T\\ 0\\ 0_T \end{array}\right] ,\left[\begin{array}{ccc}
	 	 I_T & \sigma_{x\eta}\iota_T&\sigma_{xv}\Gamma_T \\
	 		\sigma_{x\eta}\iota_T'& 1&0_T' \\
	 		\sigma_{xv}\Gamma_T'& 0_T&  I_T
	 \end{array}\right]\right)
	\end{equation}
where $0_m$ denotes an $m$-vector of zeros, $I_m$ the $(m\times m)$ identity matrix, $\iota_m$ an $m$-vector of ones, and $\Gamma_m$ an $m\times m$ matrix with ones on the subdiagonal and zeros elsewhere, namely
	\begin{equation}
		\Gamma_m = \left[\begin{array}{cc}
	 	0_{m-1}' & 0\\
	 	I_{m-1} & 0_{m-1}
	 \end{array}\right].
	\end{equation}
Under this covariance matrix structure, $\eta_i$ and $v_{i}$ are uncorrelated with each other, but both are correlated with $x_{i}$: $E[x_{it}\eta_i]=\sigma_{x\eta}$ and $x_{it}$ is predetermined but not strictly exogenous with respect to $v_{it}$. Specifically, $E[x_{it}v_{it-1}]=\sigma_{xv}$, while $E[x_{it}v_{is}]=0$ for $s\neq t-1$. We initialize the presample observations $y_{i0}$ to zero, the mean of their stationary distribution, and generate the remaining time periods according to 
$$y_{it} = \alpha_1 y_{it-1} +\alpha_2 y_{it-2} + \beta x_{it} + \eta_i + v_{it}$$
In the simulation we take
 \[ 
 \alpha_1 = 0.4,\quad \beta = 0.5,\quad \sigma_{x\eta}=0.2,\quad \sigma_{xv}=0.1,\quad N=250,\quad T=5
 \]  
 and vary $\alpha_2$ over a grid. Each grid point is based on 1000 simulation replications.

The question is how the finite sample "MAD" (Median Absolute Deviation)\footnote{We choose to use MAD instead of MSE since there appear cases where MSE is infinity. MAD is robust to outliers, and less arbitrary than truncated MSE where researcher should decide truncation point.} of the 2SLS estimators of i) $\beta$ (short run effect of $x_{it}$ on $y_{it}$) and ii) $\beta/(1-\alpha_1-\alpha_2)$ (long run effect) changes with different parameter specification, especially $\alpha_2$.  

The following shows MAD comparisons for the short-run and long-run estimators over $\alpha_2$ values. Column LW or W show MAD of the estimators when the specification corresponding to column name is chosen always. On the contrary, column GFIC shows MAD of the estimators whose specification is chosen according to GFIC.   

\begin{table}[!hpt]
\centering
\small
\begin{tabular}{  c | c c  c | c c c  }
\hline
\hline
Parameter & \multicolumn{3}{c}{Short-run Effect} & \multicolumn{3}{c}{Long-run Effect} \\
    $\alpha_2$ &      LW  &       W   &    GFIC    & LW &     W &   GFIC\\
    \hline
 0.10  &0.231&\bf{\color{blue}0.141}& 0.173& 0.801& \bf{\color{blue}0.582}& 0.688\\
  0.11 & 0.237& \bf{\color{blue}0.156}& 0.181& 0.834& \bf{\color{blue}0.633}& 0.716\\
 0.12   &0.240& \bf{\color{blue}0.174}& 0.193& 0.850& \bf{\color{blue}0.685}& 0.752\\
 0.13   &0.238& \bf{\color{blue}0.187}& 0.201& 0.870& \bf{\color{blue}0.729}& 0.787\\
 0.14 &0.220& \bf{\color{blue}0.198}& 0.203& 0.870& \bf{\color{blue}0.764}& 0.808\\
\bf{\color{red} 0.15}  & \bf{\color{blue}0.201}& 0.219& 0.211& 0.844& \bf{\color{blue}0.822}& 0.839\\
\bf{\color{red} 0.16}  & \bf{\color{blue}0.205}& 0.223& 0.210& 0.883& \bf{\color{blue}0.856}& 0.862\\
 0.17  &\bf{\color{blue}0.181}& 0.242& 0.204& \bf{\color{blue}0.860}& 0.911& 0.897\\
 0.18  & \bf{\color{blue}0.162}& 0.258& 0.189& \bf{\color{blue}0.835}& 0.959& 0.891\\
 0.19  &\bf{\color{blue}0.161}& 0.265& 0.181& \bf{\color{blue}0.866}& 0.997& 0.917\\
 0.20  &\bf{\color{blue}0.143}& 0.288& 0.162& \bf{\color{blue}0.858}& 1.054& 0.910 \\
\hline
 \hline
\end{tabular}
\vspace{0.1in}
\caption{MAD Comparison of Short-run and Long-run Estimators ($\alpha_1 = 0.4$)}
\end{table}		

As expected, specification LW including $y_{it-2}$ renders lower MAD compared to specification W as $\alpha_2$ increases. Interestingly, there are cases where different specification is favored depending on the time horizon of interest given $\alpha_2$. For $\alpha_2 = 0.15$ and 0.16, we can see that LW gives lower MAD for short-run effect whereas it gives higher MAD for long-run effect. Throughout $\alpha_2$ values considered, GFIC works well since its MAD always lies between the best and the worst specification. 


