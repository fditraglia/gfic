%!TEX root = main.tex
\section{The GFIC}
\label{sec:GFIC}
The GFIC chooses among potentially incorrect moment conditions and parameter restrictions to minimize estimator AMSE for a scalar target parameter. 
Denote this target parameter by $\mu = \varphi(\gamma, \theta)$, where  $\varphi$ is a real-valued, almost surely continuous function of the underlying model parameters $\theta$ and $\gamma$.
Let $\mu_n = \varphi(\gamma_n,\theta_0)$ and define $\mu_0$ and $\widehat{\mu}(b,c)$  analogously.
By Theorem \ref{thm:asymp} and the delta method, we have the following result.
\begin{cor} Under the hypotheses of Theorem \ref{thm:asymp},
		$$\sqrt{n}\left(\widehat{\mu}(b,c) - \mu_0 \right) \overset{d}{\rightarrow} -\nabla_\beta\varphi_0'\Xi_b '  K(b,c)\Xi_c \left(\mathscr{N}+ \left[ \begin{array}{c} 0\\ \tau\end{array}\right] - F_\gamma\delta\right) $$
where $\varphi_0 = \varphi(\gamma_0,\theta_0)$.
\end{cor}
The true value of $\mu$, however,  is $\mu_n$ rather than $\mu_0$ under Assumption \ref{assump:local}.
Accordingly, to calculate AMSE we recenter the limit distribution as follows.
\begin{cor} 
\label{cor:mulimit}
Under the hypotheses of Theorem \ref{thm:asymp},
		$$\sqrt{n}\left(\widehat{\mu}(b,c) - \mu_n \right) \overset{d}{\rightarrow} -\nabla_\beta\varphi_0'\Xi_b '  K(b,c)\Xi_c \left(\mathscr{N}+ \left[ \begin{array}{c} 0\\ \tau\end{array}\right] - F_\gamma\delta\right) -\nabla_\gamma \varphi_0' \delta$$
where $\varphi_0 = \varphi(\gamma_0,\theta_0)$.
\end{cor}

We see that the limiting distribution of $\widehat{\mu}(b,c)$ is not, in general, centered around zero: both $\tau$ and $\delta$ induce an asymptotic bias. 
Note that, while $\tau$ enters the limit distribution only once, $\delta$ has two distinct effects. 
First, like $\tau$, it shifts the limit distribution of $\sqrt{n}f_n(\gamma_0, \theta_0)$ away from zero, thereby influencing the asymptotic behavior of $\sqrt{n}\left(\widehat{\mu}(b,c) - \mu_0 \right)$. 
Second, unless the derivative of $\varphi$ with respect to $\gamma$ is zero at $(\gamma_0, \theta_0)$, $\delta$ induces a second source of bias when $\widehat{\mu}(b,c)$ is recentered around $\mu_n$. 
Crucially, this second source of bias exactly cancels the asymptotic bias present in the limit distribution of $\widehat{\gamma}_v$. 
Thus, the valid estimator of $\mu$ is asymptotically unbiased and its AMSE equals its asymptotic variance.
\begin{cor}
\label{cor:muvalid}
Under the hypotheses of Theorem \ref{thm:asymp},
	$$\sqrt{n}\left( \widehat{\mu}_v - \mu_n\right) \overset{d}{\rightarrow} -\nabla_\beta \varphi(\gamma_0, \theta_0)' K_v\mathscr{N}_g$$
where $\widehat{\mu}_v = \varphi(\widehat{\gamma}_v,\widehat{\theta}_v)$. Thus, the valid estimator $\widehat{\mu}_v$ shows no asymptotic bias and has asymptotic variance $\nabla_\beta \varphi(\gamma_0, \theta_0)'K_v \Omega_{gg}K_v'\nabla_\beta \varphi(\gamma_0, \theta_0)$.
\end{cor}
 
Using Corollary \ref{cor:mulimit}, the AMSE of $\widehat{\mu}(b,c)$ is as follows,
	\begin{align}
	\label{eq:AMSE}
		\mbox{AMSE}\left(\widehat{\mu}(b,c)\right) &= \mbox{AVAR}\left(\widehat{\mu}\left(b,c\right)\right)  + \mbox{BIAS}\left(\widehat{\mu}\left(b,c\right)\right)^2\\
		\mbox{AVAR}\left(\widehat{\mu}\left(b,c\right)\right) &= \nabla_\beta\varphi_0'\Xi_b '  K(b,c)\Omega_c K(b,c)'\Xi_b\nabla_\beta\varphi_0\\
		\mbox{BIAS}\left(\widehat{\mu}\left(b,c\right)\right) &= -\nabla_{\beta} \varphi_0' M(b,c)\left[\begin{array}{c} \delta \\ \tau\end{array} \right]
		\label{eq:bias}
\end{align}
where
\begin{equation}
	M(b,c) = \Xi_b'K(b,c) \Xi_c \left[\begin{array}{cc} -G_\gamma & 0 \\ -H_\gamma & I \end{array} \right] +\left[\begin{array}{ll} I_r & 0_{r\times q} \\ 0_{p\times r} & 0_{s\times q} \end{array} \right]
\end{equation}

The idea behind the GFIC is to construct an estimate $\widehat{\mbox{AMSE}}\left(\widehat{\mu}(b,c)\right)$ and choose the specification $(b^*,c^*)\in\mathcal{BC}$ that makes this quantity as small as possible. 
As a side-effect of the consistency of the estimators $\widehat{\beta}(b,c)$, the usual sample analogues provide consistent estimators of $K(b,c)$ and $F_{\gamma}' = (G_\gamma', H_\gamma')$ under Assumption \ref{assump:local}, and $\varphi(\widehat{\theta}_v,\gamma_0)$ is consistent for $\varphi_0$. 
Consistent estimators of $\Omega$ are also readily available under local mis-specification although the best choice may depend on the situation.\footnote{We discuss this in more detail for our examples in Section \ref{sec:examples} below.}
Since $\gamma_0$ is known, as are $\Xi_b$ and $\Xi_c$, only $\delta$ and $\tau$ remain to be estimated. 
Unfortunately, neither of these quantities is consistently estimable under local mis-specification. 
Intuitively, the data become less and less informative about $\tau$ and $\delta$ as the sample size increases since each term is divided by $\sqrt{n}$. 
Multiplying through by $\sqrt{n}$ counteracts this effect, but also stabilizes the variance of our estimators. 
Hence, the best we can do is to construct \emph{asymptotically unbiased} estimators of $\tau$ and $\delta$. 
Corollary \ref{cor:valid} provides the required estimator for $\delta$, namely $\widehat{\delta} = \sqrt{n}\left(\widehat{\gamma}_v - \gamma_0\right)$, while Lemma \ref{lem:tauhatasymp} provides an asymptotically unbiased estimator of $\tau$ by plugging $\widehat{\beta}_v$ into the sample analogue of the $h$-block of moment conditions.

\begin{cor} 
\label{cor:deltahat}
Under the hypotheses of Theorem \ref{thm:asymp},
	$\widehat{\delta} = \sqrt{n}\left(\widehat{\gamma}_v - \gamma_0\right) \overset{d}{\rightarrow} \delta - K_{v}^{\gamma} \mathscr{N}_g$
where 
$K_v = \left[G'W_{gg}G\right]^{-1}G'W_{gg}$.
%= \left(K_{v}^{\gamma '}, K_{v}^{\theta '}\right)'$. 
Hence, $\widehat{\delta}$ is an asymptotically unbiased estimator of $\delta$.
\end{cor} 

\begin{lem}
\label{lem:tauhatasymp}
Under the hypotheses of Theorem \ref{thm:asymp},
	$$\widehat{\tau} = \sqrt{n}h_n\left(\widehat{\beta}_v\right) \overset{d}{\rightarrow} \tau - HK_v \mathscr{N}_g + \mathscr{N}_h$$
where $K_v = \left[G'W_{gg}G\right]^{-1}G'W_{gg}$. Hence, $\widehat{\tau}$ is an asymptotically unbiased estimator of $\tau$.
\end{lem}

Combining Corollary \ref{cor:deltahat} and Lemma \ref{lem:tauhatasymp}, gives the joint distribution of $\widehat{\delta}$ and $\widehat{\tau}$.

\begin{thm}
\label{thm:jointbias}
Under the hypotheses of Theorem \ref{thm:asymp},
	\[\left[\begin{array}{c}\widehat{\delta}\\ \widehat{\tau}\end{array}\right] = \sqrt{n}\left[\begin{array}{c}\left(\widehat{\gamma}_v-\gamma_0\right)\\h_n(\widehat{\beta}_v)\end{array}\right]
  \overset{d}{\rightarrow} \left[\begin{array}{c}\delta\\ \tau\end{array} \right] +\Psi \mathscr{N}, \quad
  \Psi = \left[\begin{array}{cc} -K_{v}^\gamma&\mathbf{0} \\ -HK_v&I\end{array}\right]
\]
	where $K_v = \left[G'W_{gg}G\right]^{-1}G'W_{gg}$  is partitioned according to $K_v' = (K_v^{\gamma '}, K_v^{\theta '})$.
\end{thm}
Now, we see immediately from Equation \ref{eq:bias} that
$$\mbox{BIAS}\left(\widehat{\mu}\left(b,c\right)\right)^2 = \nabla_\beta \varphi_0' M(b,c) \left[\begin{array}{cc}  \delta \delta'& \delta \tau'\\ \tau \delta'& \tau \tau'\end{array}\right] M(b,c)' \nabla_\beta \varphi_0$$
Thus, the bias parameters $\tau$ and $\delta$ enter the AMSE expression in Equation \ref{eq:AMSE} as outer products: $\tau\tau'$, $\delta\delta'$ and $\tau\delta'$.
Although $\widehat{\tau}$ and $\widehat{\delta}$ are asymptotically unbiased estimators of $\tau$ and $\delta$, it does \emph{not} follow that $\widehat{\tau}\widehat{\tau}$, $\widehat{\delta}\widehat{\delta}$ and $\widehat{\tau}\widehat{\delta}'$ are  asymptotically unbiased estimators of $\tau\tau'$, $\delta\delta'$, and $\tau\delta'$. 
The following result shows how to adjust these quantities to provide the required asymptotically unbiased estimates. 

\begin{cor}
\label{cor:biasestimators}
Suppose that $\widehat{\Psi}$ and $\widehat{\Omega}$ are consistent estimators of $\Psi$ and $\Omega$. Then, $\widehat{B}$ is an asymptotically unbiased estimator of $B$, where
\[
  \widehat{B} = \left[\begin{array}{cc}  \widehat{\delta} \widehat{\delta}'& \widehat{\delta} \widehat{\tau}'\\ \widehat{\tau} \widehat{\delta}'& \widehat{\tau} \widehat{\tau}'\end{array}\right] - \widehat{\Psi} \widehat{\Omega} \widehat{\Psi}', \quad B = 
\left[\begin{array}{cc}  \delta \delta'& \delta \tau'\\ \tau \delta'& \tau \tau'\end{array}\right].
\]
\end{cor}
Combining Corollary \ref{cor:biasestimators} with consistent estimates of the remaining quantities yields the GFIC, an asymptotically unbiased estimator of the AMSE of our estimator of a target parameter $\mu$ under each specification $(b,c)\in \mathcal{BC}$
\begin{equation}
\mbox{GFIC}(b,c) =\nabla_\beta \widehat{\varphi}_0' \left[\Xi_b' \widehat{K}(b,c)\widehat{\Omega}_c \widehat{K}(b,c)'\Xi_b +  \widehat{M}(b,c) \;\widehat{B} \; \widehat{M}(b,c)'\right]\nabla_\beta \widehat{\varphi}_0.
\label{eq:GFIC}
\end{equation}
We choose the specification $(b^*,c^*)$ that minimizes the GFIC over the candidate set $\mathcal{BC}$. 

