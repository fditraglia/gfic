%!TEX root = main.tex
This paper proposes a criterion for simultaneous GMM model and moment selection: the generalized focused information criterion (GFIC). 
Rather than attempting to identify the ``true'' specification, the GFIC chooses from a set of potentially mis-specified moment conditions and parameter restrictions to minimize the mean-squared error (MSE) of a user-specified target parameter.
The intent of the GFIC is to formalize a situation common in applied practice.
An applied researcher begins with a set of fairly weak ``baseline'' assumptions, assumed to be correct, and must decide whether to impose any of a number of stronger, more controversial ``suspect'' assumptions that yield parameter restrictions, additional moment conditions, or both.
Provided that the baseline assumptions identify the model, we show how to construct an asymptotically unbiased estimator of the asymptotic MSE to select over these suspect assumptions: the GFIC. 
We go on to provide results for post-selection inference and model averaging that can be applied both to the GFIC and various alternative selection criteria.
We specialize the GFIC to three panel data examples: choosing between random and fixed effects estimators, selecting over exogeneity assumptions and lag lengths in a dynamic panel model, and choosing between pooled and mean-group estimators in the presence of heterogeneous effects. 
In the random versus fixed effects example, we also propose a novel averaging estimator that optimally weights the two estimators.
The GFIC performs well in simulations. 
We conclude by applying the GFIC to a dynamic panel data model for the price elasticity of cigarette demand. 

