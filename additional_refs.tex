\documentclass[11pt]{article}

\textwidth 16.5cm
\evensidemargin 0mm
\oddsidemargin0mm
\topmargin-18mm
\textheight 23.0cm

%----------------------------------------------------------------------------------------------------------
% PACKAGES
%----------------------------------------------------------------------------------------------------------
%\usepackage{epsfig}
%\usepackage{draftwatermark}
\usepackage{everypage}
\usepackage{lscape}
\usepackage{amsfonts}
\usepackage{amssymb}
\usepackage{amsmath}
\usepackage{setspace}
\usepackage[stable]{footmisc}
\doublespacing
%\usepackage[bottom]{footmisc}
\usepackage{float}
\usepackage{longtable}
\usepackage[center,bf]{caption}
\usepackage[debug,citecolor=blue]{hyperref}
\usepackage{theorem}
\usepackage[toc,page,title,titletoc,header]{appendix}
\usepackage{tikz}                                             % you only need this when using TikZ graphics
%\usepackage{pdfsync}
\usepackage{graphicx}
\usepackage{float}
\usepackage[final]{pdfpages}
%\usepackage{epsfig}
%\usepackage{3parttable}
%\usepackage{dcolumn}
\DeclareGraphicsExtensions{.png, .pdf, .jpg, .mps, .ps, .pdf, .eps}
\usepackage{pdfpages}
\usepackage{color}
\usepackage{xcolor}
\usepackage{rotating}
\usepackage{multirow}
\usepackage{placeins}

\usepackage{bbm}

\graphicspath{{./figuras/}}

%----------------------------------------------------------------------------------------------------------
% BIBLIOGRAPHY
%----------------------------------------------------------------------------------------------------------
\usepackage{natbib}
%{\bibliographystyle{econometrica}
\renewcommand{\cite}{\citet}

%--------
%%WATER MARK
%\SetWatermarkFontSize{1cm}
%\SetWatermarkScale{2}
%\SetWatermarkLightness{.9}
%\SetWatermarkText{Preliminary Draft}

%--------

%%%%%%%%% NEW COMMAND MACRO %%%%%%%%%%%%%%%%%%%%%%%%%%%%%%

\newcommand{\re}[1]{\smallskip\textsf{\textbf{Respuesta}} \begin{sf}\\
#1 \end{sf} \bigskip}
\newcommand{\pr}[2]{\frac{\partial #1}{\partial #2}}

\newcommand{\notorth}{\ensuremath{\perp\!\!\!\!\!\!\diagup\!\!\!\!\!\!\perp}}%
\newcommand{\orth}{\ensuremath{\perp\!\!\!\perp}}%

%%%%%%%%%%%%%%%%%%%%%%%%%%%%%%%%%%%%%%%%%%%%%%%%%%%%%%%%%%

%\input{ee.sty}

\setcounter{MaxMatrixCols}{10}

\newtheorem{example}{Example}[section]
\newtheorem{assume}{Assumption}
\newtheorem{proposition}{Proposition}[section]
\newtheorem{lemma}{Lemma}[section]
\newtheorem{corollary}{Corollary}[section]
\newtheorem{theorem}{Theorem}[section]
\newtheorem{definition}{Definition}[section]
\setcounter{section}{0}
\makeindex

\newcommand{\notindep}{\ensuremath{\perp\!\!\!\!\!\!\diagup\!\!\!\!\!\!\perp}}
\newcommand{\indep}{\ensuremath{\perp\!\!\!\perp}}
\newcommand{\Perp}{\perp \! \! \! \perp}

\newcommand{\prob}{\mathbb{P}}
\newcommand{\expec}{\mathbb{E}}


\def\inprobHIGH{\,{\buildrel p \over \longrightarrow}\,} 
\def\inprob{\,{\inprobHIGH}\,} 

\def\indistHIGH{\,{\buildrel d \over \longrightarrow}\,} 
\def\indist{\,{\indistHIGH}\,} 

%----------------------------------------------------------------------------------------------------------
% HYPER-REFERENCES CONFIGURATION - PDF
%----------------------------------------------------------------------------------------------------------

\hypersetup{
  colorlinks=true,linkcolor=blue,citecolor=black, filecolor=magenta,
citebordercolor=yellow,
  linkbordercolor = magenta,
  pdftitle={SGL},
  pdfauthor={Rojas},
  pdfcreator={\LaTeX\ with package \flqq hyperref\frqq using Textmate},
  pdfsubject={achievement},
  pdfkeywords={achievement}
}

\usepackage{psfrag}
\usepackage{afterpage}
\usepackage{subfigure}
\usepackage{setspace}
\usepackage{fancyhdr}

\newcommand{\norm}[1]{\left\lVert#1\right\rVert}

\setlength\parindent{0pt}

\begin{document}
%\title{\LARGE Problem Set 7}
%\author{Minsu Chang}
%\date{\small{\today}}


 %\maketitle \thispagestyle{empty}

\doublespacing

%\setcounter{page}{0}


\section*{Additional References}

\begin{itemize}
\item From FMSC paper, the references in red are added to to the GFIC paper:
\begin{itemize}
\item "Although Hall and Peixe (2003) and \textcolor{red}{Cheng and Liao (2013)} do consider relevance,  their aim is to avoid including redundant moment conditions after consistently eliminating invalid ones.  Some other papers that propose choosing, or combining, instruments to minimize MSE include \textcolor{red}{Donald and Newey (2001), Donald et al. (2009)}, and \textcolor{red}{Kuersteiner and Okui (2010)}.  Unlike the FMSC, however, these papers consider the higher-order bias that arises from including many valid instruments...
\item "The  framework  within  which  I  study  moment  averaging  is  related  to  the  frequentist model average estimators of Hjort and Claeskens (2003).  Two other papers that consider weighting estimators based on different moment conditions are \textcolor{red}{Xiao (2010)} and \textcolor{red}{Chen et al.
(2009)}. Whereas these papers combine estimators computed using valid moment conditions to achieve a minimum variance estimator, I combine estimators computed using potentially invalid conditions with the aim of reducing estimator AMSE." 
\end{itemize}
\item Mean-group estimation references:
\begin{itemize}
\item \textcolor{red}{Pesaran and Smith (1995)}: Pooling estimator gives inconsistent estimates where as MG estimator produces consistent estimates of the average of the parameters from dynamic heterogeneous panels.
\item \textcolor{red}{Pesaran et al. (1999)}: They propose an intermediate procedure, the pooled mean group (PMG), which imposes identical long-run coefficients but allows short-run coefficients to differ across groups. (\textcolor{blue}{*** Not sure whether we need to include this reference...})
\item \textcolor{red}{Swamy (1970)}: This paper assumes that the coefficient vector is distributed across units with the same mean and the same variance-covariance matrix. This random coefficient regression model is the setup that we consider in Section 8. (Also, how to construct consistent estimators for $\sigma_\epsilon^2, \sigma_\eta^2$ is following Swamy (1970)).
\item \textcolor{red}{Pesaran and Yamagata (2008)}: This paper proposes a standardized test of slope homogeneity for panel data models where the cross section dimension could be large relative to the time series dimension. (\textcolor{blue}{*** Not sure whether we need to include this reference...})
\end{itemize}
\item Empirical example references:
\begin{itemize}
\item \textcolor{red}{Baltagi et al. (2000)}
\end{itemize}
\end{itemize}


\end{document}
