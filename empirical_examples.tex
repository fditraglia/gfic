%!TEX root = main.tex
\section{Empirical Example: Demand for Cigarettes}
\label{sec:cigarettes}
We now consider an empirical example illustrating the GFIC in the dynamic panel setting introduced in Section \ref{sec:Dpanel}.
Out exercise is based on \cite{BaltagiEtAl2000} who study the demand for cigarettes using panel data for 46 U.S.\ states between 1963 and 1992. 
Their model is
\[
  \ln C_{it} =  \gamma \ln C_{i,t-1} + \theta \ln P_{it} + \beta_1 \ln Y_{it} + \beta_2 \ln Pn_{it} + \eta_i + \lambda_t + v_{it}
\] 
where $C_{it}$ is the number of packs of cigarettes sold per person aged 16 and above, $P_{it}$ is the real average retail price of a pack of cigarettes, $Y_{it}$ is per capita disposable income, $Pn_{it}$ is the minimum average price of a pack of cigarettes in any state that neighbors state $i$, $\eta_i$ is a state fixed effect, and $\lambda_t$ is a time fixed effect.
The lagged dependent variable in this model is meant to capture habit-persistence in cigarette consumption but it is the price elasticity not the habit-persistence \emph{per se} that is of primary interest. 

\cite{BaltagiEtAl2000} consider an exhaustive list of possible estimators for two target parameters, the short-run price elasticity $\theta$ and the long-run price elasticity $\theta/(1 - \gamma)$, and explore how the resulting estimates vary.
Here we consider selecting between four alternative estimators of the short-run price elasticity $\theta$, as in the second simulation experiment from Section \ref{sec:Dpanel_sim}.
Each specification is estimated by TSLS in first differences, using the expressions from section \ref{sec:Dpanel}.\footnote{Say how we try to be as similar as possible to Baltagi et al but there are a couple of differences. Explain how these specifications are similar to some results in Baltagi and how they're not!}
For simplicity, we assume that the controls $\ln Y_{it}$ and $\ln Pn_{it}$ are exogenous with respect to $v_{it}$.
We focus on two questions.
First: exogeneity assumption should we impose on $\ln P_{it}$?
Second: should we allow for habit persistence model by estimating $\gamma$?

Our baseline specification, $\text{LP}$, estimates estimates both $\gamma$ and $\theta$ and assumes only that $\ln P_{it}$ is predetermined with respect to $v_{it}$. 
This estimator uses the instrument set $\mathbf{z}_{it}(\ell, \text{P})$ with $\ell = 1$ from Equation \ref{eq:Zdpanel}.
For the purposes of this exercise we assume that $\text{LP}$ is correctly specified.
The specification $\text{LS}$ also estimates $\gamma$, but uses the expanded instrument set $\mathbf{z}_{it}(\ell, \text{S})$ with $\ell=1$ from equation \ref{eq:Zdpanel}.
The additional instruments used in $\text{LS}$ are only valid if $\ln P_{it}$ is strictly exogenous with respect to $v_{it}$.
Like $\text{LP}$ and $\text{LS}$, the specifications $\text{P}$ and $\text{S}$ differ in whether or not they impose that $\ln P_{it}$ is predetermined or strictly exogenous.
In contrast, however, they set $\gamma = 0$ and estimate a model with no habit persistence ($\ell =0$).
This increases the number of time periods available for estimation.

Estimates and GFIC results for all specifications appear in Table \ref{tab:cigarettes}.
In Table \ref{tab:cigarettesShort}, we use data from 1975--1980 only ($T=6$).
After first-differencing, this leaves 4 time periods for estimation in specifications that include a lag ($\text{LP}$ and $\text{LS}$) versus 5 for those that do not ($\text{P}$ and $\text{S}$).
We choose to artificially shorten the time dimension of the panel to illustrate a key feature of the GFIC, namely that it explicitly considers the number of available time periods when selecting over parameter restrictions and moment conditions. 


\begin{table}[htbp]
  \centering
    \begin{subtable}[h]{0.45\textwidth}
        \centering
     \caption{1975--1980 ($T=6$)}
     \label{tab:cigarettesShort}
     \begin{tabular}{lrrrr}\hline\hline 
         & \multicolumn{1}{c}{$\text{LP}$} & \multicolumn{1}{c}{$\text{LS}$} 
          & \multicolumn{1}{c}{$\text{P}$} & \multicolumn{1}{c}{$\text{S}$}\\
        \hline
        $\widehat{\mu}$ & -0.68 & \textcolor{blue}{\textbf{-0.32}} &  -0.28 &  -0.37\\
        Var.\ & 0.16 &  0.02 & 0.07 & 0.01\\ 
        Bias$^2$ & \multicolumn{1}{c}{---} & -4.20 & 0.01 & -3.56 \\
        GFIC  & 0.16 & -4.18 & 0.08  & -3.54  \\
        GFIC+  & 0.16 &  0.02 & 0.08  & 0.33    \\
        \hline
      \end{tabular}
    \end{subtable}
    ~
    \begin{subtable}[h]{0.45\textwidth}
      \centering
     \caption{1975--1985 ($T=11$)}
     \label{tab:cigarettesLong}
     \begin{tabular}{lrrrr}\hline\hline 
         & \multicolumn{1}{c}{$\text{LP}$} & \multicolumn{1}{c}{$\text{LS}$} 
          & \multicolumn{1}{c}{$\text{P}$} & \multicolumn{1}{c}{$\text{S}$}\\
        \hline
        $\widehat{\mu}$ & \textcolor{blue}{\textbf{-0.30}} & -0.26 &  -0.38 &  -0.28\\
        Var.\ & 0.06 & 0.01 & 0.05 & 0.01\\ 
        Bias$^2$ & \multicolumn{1}{c}{---} & 2.21 & 0.03 & 1.29\\
        GFIC  &0.06 & 2.22 & 0.08 &  1.30\\
        GFIC+  & 0.06 & 2.22 & 0.08  & 1.30  \\
           \hline
      \end{tabular}
    \end{subtable}
    \caption{Estimates and GFIC values for the price elasticity of demand for cigarettes example from Section \ref{sec:cigarettes} under four alternative specifications. Panel (a) presents results using data from 1975--1980 while Panel (b) presents results using data from 1975--1985. In each panel, the parameter estimate from the GFIC-minimizing specification is given in bold. GFIC+ gives an alternative version of the GFIC in which a negative squared bias estimate is set equal to zero.}
    \label{tab:cigarettes}
\end{table}

instrument set.
In contrast, $\text{LS}$ estimates both parameters using the ``strict exogeneity'' instrument set.
The specifications $\text{P}$ and $\text{S}$ set $\gamma=0$ and estimate only $\gamma$, using the predetermined and strictly exogenous instrument sets, respectively.
Equation \ref{eq:Zdpanel} with the instrument sets $\mathbf{z}_{it}(\ell, \text{P})$ versus $\mathbf{z}_{it}(\ell, \text{S})$ with lag length $\ell$.

\[
\Delta \ln C_{it} = \gamma \Delta \ln C_{i,t-1} +  \theta \Delta \ln P_{it} +\alpha \Delta W_{it} +  \Delta v_{it}
\]
 Suppose we are interested in whether to use a dynamic specification of cigarette demand to estimate price elasticity. Then we can project out $\Delta W_{it}$ to rewrite the model as
\[
\widetilde{C}_{it} = \gamma \widetilde{C}_{i,t-1} + \theta \widetilde{P}_{it} + \widetilde{v}_{it} 
\] 
 
 where $\widetilde{C}_{it}$ and $\widetilde{P}_{it}$ are the residuals of $\Delta \ln C_{it}$ and $\Delta \ln P_{it}$ respectively, after projecting out $\Delta W_{it}$.
 
The model selection decision is whether or not to set $\gamma = 0$. The moment selection is whether or not to use $\ln P_{it}$ (after projecting out $\Delta W_{it})$ as an instrument for period $t$.  There are four specifications:  LW, LS, W, and S.\footnote{The specification LW we consider is different from FD-2SLS in Baltagi et al. (2000). Baltagi et al. (2000) consider the following specification without time dummies:
 \[
\Delta \ln C_{it} = \beta_1 \Delta \ln C_{i,t-1} +  \beta_2 \Delta \ln P_{it} +\beta_3 \Delta \ln Pn_{it} +  \beta_4 \Delta \ln Y_{it} + \Delta v_{it}.
\]
Furthermore, they instrument $\Delta \ln C_{i,t-1}$ by the lagged values of exogenous variables $\ln P_{it}, \ln Pn_{it}, \ln Y_{it},$ $\ln P_{i,t-1}, \ln Pn_{i,t-1}, \ln Y_{i,t-1}, \ln P_{i,t-2}, \ln Pn_{i,t-2}, \ln Y_{i,t-2}$. Instead, our specification LW use instrument $\ln C_{i, t-2}$ for $\Delta \ln C_{i,t-1}$ as suggested in Anderson and Hsiao (1982).}
 Both LW and LS include a lagged dependent variable. LW and W designate the letter W for weak exogeneity assumption. Our aim is to use the GFIC to choose between competing 2SLS estimators of $\theta$ on the basis of AMSE. 
