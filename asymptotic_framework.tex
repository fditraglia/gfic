%!TEX root = main.tex
\section{Asymptotic Framework}
\todo[inline]{Need to add an identification condition!}
\label{sec:asymp}
Let $f(\cdot, \cdot)$ be a $(p+q)$-vector of moment functions of a random vector $Z$ and an $(r+s)$-dimensional parameter vector $\beta$.
To represent moment selection, we partition the moment functions according to $f(\cdot,\cdot) = \left(g(\cdot, \cdot)', h(\cdot, \cdot)'\right)'$ where $g(\cdot, \cdot)$ and $h(\cdot, \cdot)$ are $p$- and $q$-vectors.
The moment condition associated with $g(\cdot, \cdot)$ is assumed to be correct, while that associated with $h(\cdot,\cdot)$ is locally mis-specified.  
The moment selection problem is to choose which, if any, of the elements of $h$ to use in estimation. 
To represent model selection, we partition the full parameter vector according to $\beta = \left(\gamma', \theta'\right)'$, where $\gamma$ is an $r$-vector and $\theta$ an $s$-vector of parameters.
The model selection problem is to decide which if any of the elements of $\gamma$ to estimate, and which to set equal to the corresponding elements of $\gamma_0$, an $r$-vector of known constants.
The parameters contained in $\theta$ are those that we always estimate, the ``protected'' parameters.
Any specification that does not estimate the full parameter vector $\beta$ is locally mis-specified.
The precise form of the local mis-specification, over parameter restrictions and moment conditions, is as follows.

\begin{assump}[Local Mis-specification]
\label{assump:local}
Let $\{Z_{ni}\colon 1\leq i \leq n, n =1, 2, \hdots\}$ be an iid triangular array of random vectors defined on a probability space $(\Upsilon, \mathcal{F},P)$ satisfying
	\begin{enumerate}[(a)]
    \item $E[g(Z_{ni}, \gamma_n,\theta_0)]=0$, with $\gamma_n = \gamma_0 + n^{-1/2} \delta$
    \item $E[h\left( Z_{ni}, \gamma_n, \theta_0\right)] = \tau_n$, with $\tau_n = n^{-1/2} \tau$
		\item $\{f(Z_{ni}, \gamma_n, \theta_0)\colon 1\leq i \leq n, n = 1, 2, \hdots \}$ is uniformly integrable, and 
		\item $Z_{ni}\overset{d}{\rightarrow} Z_i$.
	\end{enumerate}
	where $\gamma_0$ is a (known) $r$-vector of parameter restrictions, $\delta$ an unknown $r$-vector of constants, and $\tau$ an unknown $q$-vector of constants.
\end{assump}

Under Assumption \ref{assump:local}, the true parameter vector $\beta_n = \left(\gamma_n', \theta_0' \right)'$, changes with sample size but converges to $\beta_0 = \left(\gamma_0', \theta_0'\right)'$ as $n\rightarrow \infty$. 
Unless some elements of $\delta$ are zero, any estimator that restricts $\gamma$ is mis-specified for fixed $n$. 
In the limit, however, the restriction $\gamma = \gamma_0$ holds. 
Similarly, for any fixed sample size $n$, the expectation of $h$ evaluated at the true parameter value $\beta_n$ depends on the unknown constant vector $\tau$, but this source of mis-specification disappears in the limit. 
Thus, under Assumption \ref{assump:local}, only estimators that use moment conditions from $g$ to estimate the full parameter vector $\beta$ are correctly specified. 
In the limit, however, \emph{every} estimator is correctly specified, regardless of which elements of $\gamma$ it restricts and which elements of $h$ it includes. 
The purpose of local mis-specification is to ensure that squared asymptotic bias is of the same order as asymptotic variance: Assumption \ref{assump:local} is a device rather than literal description of real-world data. 
Note that, by Assumption \ref{assump:local}, the limiting random variable $Z_i$ satisfies the population moment condition $E[f\left(Z_i,\gamma_0, \theta_0\right)]=0$. 
Since the $Z_i$ are assumed to have a common marginal law, we will use the shorthand $Z$ for $Z_i$ throughout.
For simplicity, and because it is the case for all examples we consider below, we assume that the triangular array from Assumption \ref{assump:local} is iid, although this is not strictly necessary.

Before defining the estimators under consideration, we require some further notation. Let $b$ be a \emph{model selection vector}, an $r$-vector of ones and zeros indicating which elements of $\gamma$ we have chosen to estimate. 
When $b = 1_r$, where $1_m$ represents an $m$-vector of ones, we estimate both $\theta$ and the full vector $\gamma$. 
When $b = 0_r$, where $0_m$ denotes an $m$-vector of zeros, we estimate only $\theta$, setting $\gamma=\gamma_0$. 
More generally, we estimate $|b|$ components of $\gamma$ and set the others equal to the corresponding elements of $\gamma_0$. 
Let $\gamma^{(b)}$ be the $|b|$-dimensional subvector of $\gamma$ corresponding to those elements selected for estimation. 
Similarly, let $\gamma^{(-b)}_0$ denote the $(r-|b|)$-dimensional subvector containing the values to which we set those components of $\gamma$ that are \emph{not} estimated.
Analogously, let $c=\left(c_g', c_h'\right)'$ be a \emph{moment selection vector}, a $(p+q)$-vector of ones and zeros indicating which of the moment conditions we have chosen to use in estimation. 
We denote by $|c|$ the total number of moment conditions used in estimation. 
Let $\mathcal{BC}$ denote the collection of all model and moment selection pairs $(b,c)$ under consideration.


To express moment and model selection in matrix form, we define the selection matrices $\Xi_b$ and $\Xi_c$.
Multiplying $\beta$ by the $(|b| + s)\times(r+s)$ \emph{model selection matrix} $\Xi_b$ extracts the elements corresponding to $\theta$ and the subset of $\gamma$ indicated by the model selection vector $b$. 
Thus $\Xi_b \beta = \left(\gamma^{(b)'}, \theta' \right)'$.
Similarly, multiplying a vector by the $|c|\times(p+q)$ moment selection matrix $\Xi_c$ extracts the components corresponding to the moment conditions indicated by the moment selection vector $c$. 

To express the estimators themselves, define the sample analogue of the expectations in Assumption \ref{assump:local} as follows,
\begin{equation}
	f_n(\beta) = \frac{1}{n}\sum_{i=1}^n f(Z_{ni}, \gamma, \theta) = \left[\begin{array}{c}g_n(\beta)\\ h_n(\beta) \end{array} \right] = \left[\begin{array}{c} n^{-1}\sum_{i=1}^n g(Z_{ni}, \gamma, \theta)\\n^{-1}\sum_{i=1}^n h(Z_{ni}, \gamma, \theta) \end{array}\right]
\end{equation}
and let $\widetilde{W}$ be a $(q+p)\times(q+p)$ positive semi-definite weighting matrix
\begin{equation}
	\widetilde{W} = \left[\begin{array}{cc} \widetilde{W}_{gg} & \widetilde{W}_{gh}\\ \widetilde{W}_{hg} & \widetilde{W}_{hh}  \end{array}\right]
\end{equation}
partitioned conformably to the partition of $f(Z,\beta)$ by $g(Z,\beta)$ and $h(Z,\beta)$. Each model and moment selection pair $(b,c)\in \mathcal{BC}$ defines a $(|b|+s)$-dimensional estimator $\widehat{\beta}(b,c)=( \widehat{\gamma}^{(b)}(b,c)',\widehat{\theta}(b,c)')'$ of $\beta^{(b)}= \left(\gamma^{(b)'}, \theta'  \right)'$ according to
	\begin{equation}
    \widehat{\beta}(b,c) = \underset{\beta^{(b)}\in \mathbf{B}^{(b)}} {\mbox{arg min}}\;\left[\Xi_c f_n\left(\beta^{(b)}, \gamma_0^{(-b)}\right)\right]'\left[\Xi_c \widetilde{W}\Xi_c' \right]\left[\Xi_c f_n\left(\beta^{(b)}, \gamma_0^{(-b)}\right)\right].
	\end{equation}
A particularly important special case is the estimator using only the moment conditions in $g$ to estimate the full parameter vector $\beta = \left(\theta', \gamma' \right)'$, the \emph{valid} estimator:
\begin{equation}
  \widehat{\beta}_v =\left[\begin{array}{c}\widehat{\gamma}_{v} \\  \widehat{\theta}_{v} \end{array} \right] =\underset{\beta \in \mathbf{B}}{\mbox{arg min}}\; g_n(\beta)' \widetilde{W}_{gg} \; g_n(\beta).
\end{equation}
Because it is correctly specified both for finite $n$ and in the limit, the valid estimator contains the information we use to identify $\tau$ and $\delta$, and thus carry out moment and model selection. 
For estimation based on $g$ alone to be possible, we require $p \geq r+s$. 
This is assumed throughout.
Because Assumption \ref{assump:local} ensures that they are correctly specified in the limit, \emph{all} candidate specifications $(b,c)\in \mathcal{BC}$ provide consistent estimators of $\theta_0$ under standard, high level regularity conditions (see Assumption \ref{assump:high-level}).
Essential differences arise, however, when we consider their respective asymptotic distributions. 
Let
		\begin{equation}F  =  \left[\begin{array}{cc}  \nabla_{\gamma'}g\left(Z,\gamma_0, \theta_0\right) &   \nabla_{\theta'}g\left(Z,\gamma_0, \theta_0\right) \\
 \nabla_{\gamma'}h\left(Z,\gamma_0, \theta_0\right)  &   \nabla_{\theta'}h\left(Z,\gamma_0, \theta_0\right)  
		 \end{array}\right] 
	\end{equation}
partitioned according to
	\begin{equation}F  = \left[\begin{array}{cc}F_\gamma & F_\theta \end{array}\right] = 
		  \left[\begin{array}{cc} G_\gamma & G_\theta \\
					 H_\gamma& H_\theta
		 \end{array}\right] = \left[\begin{array}{c} G\\ H \end{array}\right]
	\end{equation}
and define
	\begin{equation}
		\Omega = Var\left[ \begin{array}{c} g(Z, \gamma_0, \theta_0) \\ h(Z, \gamma_0, \theta_0) \end{array}\right] =\left[ \begin{array}{cc}\Omega_{gg} & \Omega_{gh}\\ \Omega_{hg}& \Omega_{hh} \end{array} \right].
	\end{equation}
Notice that each of these expressions involves the limiting random variable $Z$ rather than $Z_{ni}$. 
Thus, the corresponding expectations are taken with respect to a distribution for which all moment conditions have expectation zero  evaluated at $(\gamma_0,\theta_0)$.
Finally, let $F(b,c) = \Xi_c F \Xi_b'$ and similarly define $\Omega_c = \Xi_c \Omega \Xi_c'$ and $W_c = \Xi_c W \Xi_c'$ where $W$ is the positive definite probability limit of $\widetilde{W}$. 
Under Assumption \ref{assump:local}, both $\delta$ and $\tau$ induce a bias term in the limiting distribution of $\sqrt{n}\left(\widehat{\beta}(b,c) - \beta_0^{(b)}\right)$. 
The key result is as follows.
\begin{assump}[High-level Regularity Conditions]
  \label{assump:high-level}
    \mbox{}
\begin{itemize}
\item[(a)] $\beta_0$ lies in the interior of $\Theta$, a compact set
\item[(b)] $\widetilde{W} \rightarrow_{p} W$, a positive definite matrix
\item[(c)] $\widetilde{W}_c \Xi_c E[f(Z, \beta)] = 0$ if and only if $\beta = \beta_0$, where $W_c = \Xi_c W \Xi_c'$
\item[(d)] $E[f(Z,\beta)]$ is continuous on $\Theta$
\item[(e)] $sup_{\beta\in \Theta} ||f_n(\beta) - E[f(Z,\beta)]||\rightarrow_p 0$
\item[(f)] $f$ is Z-almost surely differentiable in an open neighborhood $\mathcal{B}$ of $\beta_0$
\item[(g)] $sup_{\beta \in \Theta} ||\nabla_{\beta} f_n(\beta) - F(\beta)|| \rightarrow_p 0 $
\item[(h)] $\sqrt{n}f_n(\beta_0) - \sqrt{n}E[f(Z,\beta_0)] \rightarrow_d \mathscr{N}$ where $\mathscr{N} \sim N_{p+q}(0,\Omega)$
\item[(i)] $F(b,c)' W_c F(b,c)$ is invertible, where $W_c = \Xi_c W \Xi_c'$
\end{itemize}
\end{assump}
\begin{thm}[Asymptotic Distribution]
\label{thm:asymp}
Under Assumptions \ref{assump:local}--\ref{assump:high-level}
		\begin{equation}
		\sqrt{n}\left(\widehat{\beta}(b,c) - \beta_0^{(b)}\right) \overset{d}{\rightarrow} - K(b,c)\Xi_c \left(\mathscr{N}+ \left[ \begin{array}{c} 0\\ \tau\end{array}\right] - F_\gamma\delta\right)
	\end{equation}
where $\beta_0^{(b)'} = (\theta_0, \gamma_0^{(b)})$, $K(b,c) = \left[F(b,c)'W_c F(b,c)\right]^{-1} F(b,c)' W_c$
and
	\begin{equation}
		\mathscr{N} = \left[\begin{array}{c} \mathscr{N}_g\\  \mathscr{N}_h\end{array}\right]\sim \mbox{N}\left(\left[\begin{array}{c} 0\\ 0\end{array}\right] , \left[ \begin{array}{cc}\Omega_{gg} & \Omega_{gh}\\ \Omega_{hg}& \Omega_{hh} \end{array} \right]\right).
	\end{equation}
\end{thm}	
Because it employs the correct specification, the valid estimator of $\theta$ shows no asymptotic bias.
Moreover, the valid estimator of $\gamma$ has an asymptotic distribution that is centered around $\delta$, suggesting an estimator of this bias parameter.
\begin{cor}[Asymptotic Distribution of Valid Estimator]
\label{cor:valid}
Under Assumptions \ref{assump:local}--\ref{assump:high-level}
		$$\sqrt{n}\left( \widehat{\beta}_v - \beta_0 \right) = \sqrt{n}\left(\begin{array}{c} \widehat{\theta}_v - \theta_0\\ \widehat{\gamma}_v - \gamma_0\end{array} \right) \overset{d}{\rightarrow}  \left[\begin{array}{c} 0\\ \delta\end{array}\right] -K_v \mathscr{N}_g$$
where $K_v = \left[G'W_{gg}G\right]^{-1}G'W_{gg}$ and $W_{gg} = \mbox{plim }\widetilde{W}_{gg}$.
\end{cor}
We use these results in the following section to construct the GFIC.

