%!TEX root = main.tex
\section{Dynamic Panel Example}
\label{sec:panel}
We now specialize the GFIC to a dynamic panel model of the form 
\begin{equation}
  y_{it} = \gamma_1 y_{it-1} + \cdots + \gamma_k y_{it-k} + \theta x_{it} + \eta_i + v_{it}
  \label{eq:truepanel}
\end{equation}
where $i = 1, \hdots, n$ indexes individuals and $t=1, \hdots, T$  indexes time periods. 
For simplicity, and without loss of generality, we suppose that there are no exogenous time-varying regressors and that all random variables are mean zero.\footnote{Alternatively, we can simply de-mean and project out any time-varying exogenous covariates after taking first-differences.} 
The unobserved error $\eta_i$ is a correlated individual effect: $\sigma_{x\eta}\equiv \mathbb{E}\left[ x_{it}\eta_i \right]$ may not equal zero. 
The endogenous regressor $x_{it}$ is assumed to be predetermined but not necessarily strictly exogenous: $\mathbb{E}[x_{it} v_{is}]=0$ for all $s \geq t$ but may be nonzero for $s < t$.  
We assume throughout that $y_{it}$ is stationary, which requires both $x_{it}$ and $u_{it}$ to be stationary and $|\boldsymbol{\gamma}| < 1$ where $\boldsymbol{\gamma} = (\gamma_1, \dots, \gamma_k)'$.
Our goal is to estimate one of the following two target parameters with minimum MSE: 
\begin{equation}
  \mu_{\text{SR}} \equiv \theta, \quad \mu_{\text{LR}} \equiv \frac{\theta}{1- (\gamma_1 + \cdots + \gamma_k)}.
  \label{eq:paneltarget}
\end{equation}
where $\mu_{\text{SR}}$ denotes the short-run effect and $\mu_{\text{LR}}$ the long-run effect of $x$ on $y$.

The question is which assumptions to use in estimation.
Naturally, the answer may depend on whether our target is $\mu_{SR}$ or $\mu_{LR}$.
Our first decision is what assumption to impose on the relationship between $x_{it}$ and $v_{it}$.
This is the \emph{moment selection} decision.
We assumed above that $x$ is predetermined.
Imposing the stronger assumption of strict exogeneity gives us more and stronger moment conditions, but using these in estimation introduces a bias if $x$ is not in fact strictly exogenous.
Our second decision is how many lags of $y$ to use in estimation.
This is the \emph{model selection} decision.
The true model contains $k$ lags of $y$.
If we estimate only $r < k$ lags we not only have more degrees of freedom but more observations: every additional lag of $y$ requires us to drop one time period from estimation. 
In the short panel datasets common in microeconomic applications, losing even one additional time period can represent a substantial loss of information.
At the same time, unless $\gamma_{r+1} = \cdots = \gamma_k = 0$, failing to include all $k$ lags in the model introduces a bias.
%In this example, the GFIC simultaneously chooses over exogeneity assumptions for $x$ and lag length for $y$ to optimally trade off bias and variance.

To eliminate the individual effects $\eta_i$ we work in first differences.
Defining $\Delta$ in the usual way, so that $\Delta y_{it} = y_{it} - y_{it-1}$ and so on, we can write Equation \ref{eq:truepanel} as
\begin{align}
  \Delta y_{it} = \gamma_1 \Delta y_{it-1} + \cdots + \gamma_k \Delta y_{it-k} + \theta \Delta x_{it} + \Delta v_{it}.
  \label{eq:truepaneldiff}
\end{align}
For simplicity and to avoid many instruments problems -- see e.g.\ \cite{Roodman} -- we focus here on estimation using the instrument sets
\begin{align}
  \mathbf{z}'_{it}(\ell, \text{P}) &\equiv \left[
  \begin{array}{cccc}
    y_{it-2} & \cdots & y_{it-(\ell + 1)} & x_{it-1}
  \end{array}
\right] & 
\mathbf{z}'_{it}(\ell,\text{S}) &\equiv \left[
\begin{array}{cc}
  \mathbf{z}_{it}'(\ell,\text{P}) & x_{it}
\end{array}
\right]
\label{eq:Zdpanel}
\end{align}
similar to \cite{AndersonHsiao}.
Modulo a change in notation, one could just as easily proceed using the instrument sets suggested by \cite{ArellanoBond}.
Throughout this discussion we use $\ell$ as a placeholder for the lag length used in estimation.
If $\ell = 0$, $\mathbf{z}'_{it}(0,\text{P}) = x_{it-1}$ and $\mathbf{z}'_{it}(0,\text{S}) = (x_{it-1}, x_{it})$.
Given these instrument sets, we have $(\ell + 1)\times (T -\ell - 1)$ moment conditions if $x$ is assumed to be predetermined versus $(\ell + 2)\times (T - \ell - 1)$ if it is assumed to be strictly exogenous, corresponding to the instrument matrices
  $Z_i(\ell,\text{P}) = \mbox{diag}\left\{ \mathbf{z}'_{it}(\ell,\text{P})  \right\}_{t = \ell + 2}^T$ and  $Z_i(\ell,\text{S}) = \mbox{diag}\left\{\mathbf{z}'_{it}(\ell,\text{S}) \right\}_{t = \ell +2}^T$.
To abstract for a moment from the model selection decision, suppose that we estimate a model with the true lag length: $\ell = k$.
The only difference between the P and S sets of moment conditions is that the latter adds over-identifying information in the form of $E[x_{it}\Delta v_{it}]$.
If $x$ is strictly exogenous, this expectation equals zero, but if $x$ is only predetermined, then $E[x_{it}\Delta v_{it}] = -E[x_{it}v_{it-1}] \neq 0$ so the over-identifying moment condition is invalid.
Given the instrument sets that we consider, this is the only violation of strict exogeneity that is relevant for our moment selection so we take $E[x_{it}v_{it-1}] = -\tau/\sqrt{n}$.

In the examples and simulations described below we consider two-stage least squares (TSLS) estimation of $\mu_{SR}$ and $\mu_{LR}$ using the instruments defined in Equation \ref{eq:Zdpanel}. 
For simplicity, we select between two lag length specifications: one of which is correct, $\ell = k$, and one of includes one lag too few: $\ell = r$ where $r = k-1$.
Accordingly, we make coefficient associated with the $k$\textsuperscript{th} lag local to zero.
Let $\boldsymbol{\gamma}' = (\gamma_1, \cdots, \gamma_{k-1}, \gamma_{k})$ denote the full vector of lag coefficients that $\boldsymbol{\gamma}_{r} = (\gamma_1, \cdots, \gamma_{r})$ the first $r = k-1$ lag coefficients.
Then, the true parameter vector is $\beta_n = (\boldsymbol{\gamma}'_{r}, \delta/\sqrt{n}, \theta)'$ which becomes, in the limit, $\beta = (\boldsymbol{\gamma}'_r, 0, \theta)'$.
To indicate the subvector of $\beta$ that excludes the $k$\textsuperscript{th} lag coefficient, let $\beta_{r} = (\boldsymbol{\gamma}_r', \theta)'$.

Because the two lag specifications we consider use different time periods in estimation, we require some additional notation to make this clear. 
First let
$\Delta \mathbf{y}_{i} = [\Delta y_{i,k+2}, \cdots, \Delta y_{iT}]'$ and $\Delta \mathbf{y}^+_{i} = [\Delta y_{i,k+1}, \Delta y_{i,k+2}, \cdots, \Delta y_{iT}]'$ 
where the superscript ``+'' indicates the inclusion of an additional time period: $t = k+1$.
Define $\Delta \mathbf{x}_i$, $\Delta \mathbf{x}_{i}^{+}$, $\Delta \mathbf{v}_i$, and $\Delta \mathbf{v}_{i}^{+}$ analogously.
Next, define $L^k\Delta \mathbf{y}_i^{+} = [\Delta y_{i1}, \Delta y_{i2}, \cdots, \Delta y_{iT-k}]'$ where $L^k$ denotes the element-wise application of the $k$\textsuperscript{th} order lag operator.
Note that the first element of $L^k\Delta \mathbf{y}_{i}^{+}$ is unobserved since $\Delta y_{i1} = y_{i1} - y_{i0}$ but $t=1$ is the first time period.
Now we define the matrices of regressors for the two specifications: 
\begin{align*}
  W_i^{+'}(r) &= \left[
  \begin{array}{ccccc}
    L \Delta \mathbf{y}_i^{+} &  L^2 \Delta \mathbf{y}_i^{+} & \cdots & L^{k-1}\Delta \mathbf{y}_i^{+} & \Delta \mathbf{x}_i^+
  \end{array}
\right]\\
  W_i'(k) &= \left[
  \begin{array}{cccccc}
    L \Delta \mathbf{y}_i &  L^2 \Delta \mathbf{y}_i & \cdots & L^{k-1}\Delta \mathbf{y}_i & L^k\Delta \mathbf{y}_i & \Delta \mathbf{x}_i
  \end{array}
\right].
\end{align*}
Note that $W_i^{+}(r)$ contains one more row than $W_i(k)$ but $W_i(k)$ contains one more column that $W_i^{+}(r)$: removing the $k$\textsuperscript{th} lag from the model by setting $\ell = r = k-1$ allows us to use an additional time period in estimation and reduces the number of regressors by one. 
Stacking over individuals, let $\Delta \mathbf{y} = [\Delta \mathbf{y}'_1 \cdots \Delta \mathbf{y}'_n]'$, $W_\ell = [W_1(\ell) \cdots W_n(\ell)]'$ and define $\Delta \mathbf{y}^{+}$ and $W_\ell^{+}$ analogously, where $\ell$ denotes the lag length used in estimation.
Finally, let $Z'(\ell,\cdot) = [Z'_1(\ell,\cdot) \cdots Z'_n(\ell,\cdot)]$ where $(\cdot)$ is $\text{P}$ or $\text{S}$ depending on the instrument set is in use.
Using this notation, under local mis-specification the true model is
\begin{align}
  \Delta \mathbf{y} &= W(k)\beta_n + \Delta \mathbf{v} &  \Delta \mathbf{y}^{+} &= W(k)^{+}\beta_n + \Delta \mathbf{v}^+
\end{align}
Using the shorthand $\widehat{Q} \equiv n[W' Z(Z'Z)^{-1} Z'W]^{-1}W'Z(Z'Z)^{-1}$ our candidate estimators are
\begin{align}
  \widehat{\beta}(k,\cdot) &= \widehat{Q}(k,\cdot)\left[ \frac{Z'(k,\cdot)\Delta \mathbf{y}}{n} \right]& 
  \widehat{\beta}(r,\cdot) &= \widehat{Q}(r,\cdot)\left[ \frac{Z'(r,\cdot)\Delta \mathbf{y}^{+}}{n} \right]
  \label{eq:DpanelEstimators}
\end{align}
where $(\cdot)$ is either $\text{P}$ or $\text{S}$ depending on which instrument set is used and $r = k-1$, one lag fewer than the true lag length $k$.
The following result describes the limit distribution of $\widehat{\beta}(k,\text{P})$, $\widehat{\beta}(k,\text{S})$, $\widehat{\beta}(r,\text{P})$, and $\widehat{\beta}(r,\text{S})$ which we will use to construct the GFIC.

\begin{thm}[Limit Distributions for Dynamic Panel Estimators]
  \label{thm:limitDpanel}
  Let $(y_{nit},x_{nit}, v_{nit})$ be a triangular array of random variables that is iid over $i$, stationary over $t$, and satisfies Equation \ref{eq:truepaneldiff} with $\gamma_k = \delta / \sqrt{n}$.
  Suppose further that $x_{it}$ is predetermined with respect to $v_{it}$ but not strictly exogenous: $E[x_{it}\Delta v_{it}] = \tau/\sqrt{n}$.
  Then, under standard regularity conditions,
  \begin{align*}
    \sqrt{n}\left[ \widehat{\beta}(k,\text{P})-\beta \right] &\rightarrow^d 
    \left[
    \begin{array}{ccc}
    \mathbf{0}_{k-1}'& \delta & 0
    \end{array}
  \right]' + 
    Q\left(k,\text{P} \right) \mbox{N}\left(\mathbf{0}, \mathcal{V}(k,\text{P})\right)  \\
    \sqrt{n}\left[ \widehat{\beta}(k,\text{S})-\beta \right] &\rightarrow^d 
    \left[
    \begin{array}{ccc}
    \mathbf{0}_{k-1}'& \delta & 0
    \end{array}
  \right]' + 
     Q\left(k,\text{S} \right) \left\{ \boldsymbol{\iota}_{T-(k +1)} \otimes \left[
    \begin{array}{c}
      \mathbf{0}_{k+1} \\ \tau
    \end{array}
  \right] + \mbox{N}\left(\mathbf{0}, \mathcal{V}(k,\text{S})\right)\right\}\\
    \sqrt{n}\left[ \widehat{\beta}(r,\text{P})- \beta_r \right] &\rightarrow^d Q(r,\text{P}) \left[\boldsymbol{\iota}_{T-k} \otimes \delta \boldsymbol{\psi}_{\text{P}} + \mbox{N}\left(\mathbf{0}, \mathcal{V}(r,\text{P}) \right) \right]\\
    \sqrt{n}\left[ \widehat{\beta}(r,\text{S})- \beta_r\right] &\rightarrow^d Q(r,\text{S}) \left[\boldsymbol{\iota}_{T-k} \otimes 
    \left( \delta \left[
  \begin{array}{c}
    \boldsymbol{\psi}_{\text{P}} \\ \psi_{\text{S}}
\end{array}
\right] + \left[
\begin{array}{c}
  \mathbf{0}_{k} \\ \tau
\end{array}
\right]\right) + \mbox{N}\left( \mathbf{0}, \mathcal{V}\left(r,\text{S}\right) \right)\right]
  \end{align*}
  where $k = r - 1$, $\beta' = (\gamma_1, \hdots, \gamma_{r}, 0, \theta)$, $\beta_r' = (\gamma_1, \hdots, \gamma_{r}, \theta)$, $\mathcal{V}(k,\cdot) = \mbox{Var}\left[ Z_i(k,\cdot) \Delta \mathbf{v}_i  \right]$, $\mathcal{V}(r,\cdot) = \mbox{Var}\left[ Z_i(r,\cdot) \Delta \mathbf{v}^{+}_i  \right]$, $\widehat{Q}(\ell,\cdot) \rightarrow_p Q(\ell,\cdot)$, $\boldsymbol{\psi}_{\text{P}} = E[\textbf{z}_{it}(r,\text{P}) \Delta y_{it -k}]$, $\psi_{\text{S}} = E[x_{it} \Delta y_{it-k}]$, $\mathbf{z}_{it}(\ell,\cdot)$ is as in Equation \ref{eq:Zdpanel}, $Z_i(\ell, \cdot)= \mbox{diag}\{\mathbf{z}_{it}'(\ell, \cdot)\}_{t=\ell+2}^T$ and $\boldsymbol{\iota}_{d}$ denotes a $d$-vector of ones.
\end{thm}

To operationalize the GFIC, we need to provide appropriate estimators of all quantities that appear in Theorem \ref{thm:limitDpanel}.
To estimate ${Q}(k,\text{P})$, ${Q}(k,\text{S})$, ${Q}(r,\text{P})$, and ${Q}(r,\text{S})$ we employ the usual sample analogues $\widehat{Q}(\cdot,\cdot)$ given above, which remain consistent under local mis-specification.
There are many consistent estimators for the variance matrices $\mathcal{V}(k,\text{P})$, $\mathcal{V}(k,\text{S})$, $\mathcal{V}(r,\text{P})$, $\mathcal{V}(r,\text{S})$ under local mis-specification.
In our simulations and empirical example below, we employ the usual heteroskedasticity-consistent, panel-robust variance matrix estimator.
Because $E[\mathbf{z}_{it}(\ell,\text{S})\Delta v_{it}]\neq 0$, we center our estimators of $\mathcal{V}(\ell, \text{S})$ by subtracting the sample analogue of this expectation when calculating the sample variance.
We estimate $\boldsymbol{\psi}_{\text{P}}$ and $\psi_{\text{S}}$ as follows
\begin{align*}
  \widehat{\boldsymbol{\psi}}_{\text{P}} &= \frac{1}{n(T - k - 1)}\sum_{t = k+2}^T \sum_{i = 1}^n \mathbf{z}_{it}(r,\text{P}) \Delta y_{it-k} &
  \widehat{\psi}_{\text{S}} &= \frac{1}{n(T - k - 1)}\sum_{t = k+2}^T \sum_{i = 1}^n x_{it} \Delta y_{it-k}
\end{align*}
using our assumption of stationarity from above.
The only remaining quantities we need to construct the GFIC involve the bias parameters $\delta$ and $\tau$. 
We can read off an asymptotically unbiased estimator of $\delta$ directly from Theorem \ref{thm:limitDpanel}, namely $\widehat{\delta} = \sqrt{n}\; \widehat{\gamma}_k(k,\text{P})$ the estimator of $\gamma_k$ based on the instrument set that assumes only that $x$ is pre-determined rather than strictly exogenous.
To construct an asymptotically unbiased estimator of $\tau$, we use the residuals from the specification that uses \emph{both} the correct moment conditions and the correct lag specification, specifically
\begin{equation}
  \label{eq:DpanelTau}
  \widehat{\tau} = \left( \frac{\boldsymbol{\iota}_{T-k-1}'}{T - k - 1} \right) n^{-1/2} X' \left[\Delta \mathbf{y} - W(k)\widehat{\beta}(k,\text{P})  \right]
\end{equation}
where $X' = [X_1 \cdots X_n]$ and $X_i = \mbox{diag}\left\{ x_{it} \right\}_{t = k + 2}^{T}$.
The following result gives the joint limiting behavior of $\widehat{\delta}$ and $\widehat{\tau}$, which we will use to construct the GFIC.
\todo[inline]{Add a sentence talking about why we take the time average in our definition of $\tau$.}

\begin{thm}[Joint Limit Distribution of $\widehat{\delta}$ and $\widehat{\tau}$]
  \label{thm:DpanelJoint}
  Under the conditions of Theorem \ref{thm:limitDpanel},
  \[
    \left[
      \begin{array}{c} 
        \widehat{\delta} - \delta \\ \widehat{\tau} - \tau 
      \end{array} 
    \right] \overset{d}{\rightarrow} \Psi \mbox{N}\left(\mathbf{0}, \Pi\,\mathcal{V}\left(k,\text{S}\right)\,\Pi'\right)
  \]
  where $\widehat{\delta} = \sqrt{n}[ \mathbf{e}_k' \,\widehat{\beta}(k,\text{P})]$, $\mathbf{e}_k = (\mathbf{0}_{k-1}', 1, 0)'$,  $\widehat{\tau}$ is as defined in Equation \ref{eq:DpanelTau}, 
\[
  \Psi = \left[
  \begin{array}{cc}
    \displaystyle\left( \frac{\boldsymbol{\iota}'_{T-k-1}}{T-k-1} \right)  \left\{ \boldsymbol{\xi}' Q(k,\text{P}) \otimes \boldsymbol{\iota}'_{T-k-1} \right\}& \displaystyle \left(\frac{\boldsymbol{\iota}_{T-k-1}}{T-k-1}\right) \\ 
    \mathbf{e}_k' Q(k,\text{P}) & \mathbf{0}'_{T-k-1}
  \end{array}
\right],
\]
  $\boldsymbol{\xi}' = E\left\{ x_{it} \left[
    \begin{array}{cccc}
      L \Delta y_{it} & \cdots & L^k \Delta y_{it} & \Delta x_{it}
  \end{array} \right]\right\}$,
  the variance matrix $\mathcal{V}(k,\text{S})$ is as defined in Theorem \ref{thm:limitDpanel}, the permutation matrix $\Pi = \left[
  \begin{array}{cc}
    \Pi_1' & \Pi_2'
  \end{array}
\right]'$ with $\Pi_1 = I_{T-k-1} \otimes \left[
\begin{array}{cc}
  I_{k+1} & \mathbf{0}_{k+1}
\end{array}
\right]$ and $\Pi_2 = I_{T-k-1}\otimes \left[
\begin{array}{cc}
  \mathbf{0}_{k+1}' & 1
\end{array}
\right]$,
  $\boldsymbol{\iota}_{d}$ is a $d$-vector of ones and $I_d$ the $(d\times d)$ identity matrix.
\end{thm}

\todo[inline]{Needs to be re-written from here down. Explain how to estimate $\xi$ and explain about the quantities needed for the different target parameters.}

As described above for the general GMM case, asymptotically unbiased estimators of $\tau$ and $\delta$ require a bias correction to provide asymptotically unbiased estimators of the quantities $\tau^2$, $\delta^2$ and $\tau\delta$ needed to estimate AMSE. 
To carry out this correction, we use the joint distribution of the bias parameter estimators:
\begin{cor}[Asymptotically Unbiased Estimators]
Asymptotically unbiased estimators of $\delta^2$, $\tau^2$ and $\tau\delta$ are given by
\begin{eqnarray}
	\delta^2 \colon &&\widehat{\delta}^2 - \widehat{\sigma}_\delta^2\\
	\tau^2 \colon &&\widetilde{\tau}^2 - \widehat{\sigma}_\tau^2\\
	\tau \delta \colon && \widetilde{\tau}\widehat{\delta} - \widehat{\sigma}_{\tau\delta}
\end{eqnarray}
where $\widehat{\sigma}_\delta^2$, $\widehat{\sigma}_\tau^2$ and $\widehat{\sigma}_{\tau\delta}$ are consistent estimators of the elements of
$$\left[\begin{array}{cc} K_{LW}^\gamma&0 \\ \left(\frac{\iota'_{T-2}}{T-2} \right) \Psi&  \left(\frac{\iota'_{T-2}}{T-2} \right) I\end{array} \right] \mathcal{V}_{LS}  \left[\begin{array}{cc} K_{LW}^\gamma&0 \\ \left(\frac{\iota'_{T-2}}{T-2} \right) \Psi&  \left(\frac{\iota'_{T-2}}{T-2} \right) I\end{array} \right]'$$
\end{cor}

\todo[inline]{Also need to explain about the derivatives needed when we want to apply the GFIC to the estimator for the long-run effect}


%%%%%%%%%%%%%%%%%%%%%%%%%%%%%%%%%%%%%%%%%%%%%%%%%%%%%%%%%%%%%%

\subsection*{(b) Estimators for Short-run Effect}


Suppose we are interested in short-run effect of $x_{it}$ on $y_{it}$, i.e. $\beta$.  Then  by $\Delta-$method,

	\begin{eqnarray}
 			\sqrt{n} (\widehat{\beta}_{LW}-\beta) &\overset{d}{\rightarrow}& N\left(0,\quad \left[\begin{array}{ccc} 0& 0& 
1 \end{array}\right] K_{LW}\mathcal{V}_{LW} K_{LW}' \left[\begin{array}{l} 0\\ 0\\ 
1 \end{array}\right]  \right)
	\end{eqnarray}
and
\begin{eqnarray}
 	\sqrt{n}(\widehat{\beta}_{W}-\beta) &\overset{d}{\rightarrow}&\left[\begin{array}{cc} 0& 1 \end{array}\right]
 	K_W  \left[\delta \left[\begin{array}{l} \psi_0\\ 
\psi_1 \end{array}\right]  \otimes \iota_{T-2} + N\left(0, \mathcal{V}_W\right) \right].
	\end{eqnarray}
	
%	We can choose between $\widehat{\beta}_{LW}$ and $\widehat{\beta}_W$ comparing their AMSE.
	
\subsection*{(c) Estimators for Long-run Effect}
\vspace{0.1in}

	Suppose we are interested in long-run effect which is expressed as $h(\theta) = h(\alpha_1, \alpha_2, \beta) = \frac{\beta}{1-\alpha_1-\alpha_2}$ and $h_w(\theta) = \frac{\beta}{1-\alpha_1}$. Note that $h'(\theta) = \left[\begin{array}{ccc} \frac{\beta}{(1-\alpha_1-\alpha_2)^2}& \frac{\beta}{(1-\alpha_1-\alpha_2)^2} & \frac{1}{1-\alpha_1-\alpha_2}\end{array}\right]_{\alpha_2=0} =\left[\begin{array}{ccc} \frac{\beta}{(1-\alpha_1)^2}& \frac{\beta}{(1-\alpha_1)^2} & \frac{1}{1-\alpha_1}\end{array}\right] $. Also, $h_w'(\theta) = \left[\begin{array}{ccc} \frac{\beta}{(1-\alpha_1-\alpha_2)^2}& \frac{1}{1-\alpha_1-\alpha_2}\end{array}\right]_{\alpha_2=0} =\left[\begin{array}{ccc} \frac{\beta}{(1-\alpha_1)^2}& \frac{1}{1-\alpha_1}\end{array}\right] $. From $\Delta-$method, we can get
	
	\begin{eqnarray}
 			\sqrt{n} (h(\widehat{\theta}_{LW})-h(\theta)) &\overset{d}{\rightarrow}&  			N\left(0,\,\, h'(\theta)K_{LW}\mathcal{V}_{LW}K_{LW}'h'(\theta) \right)
	\end{eqnarray}
	
	
and
\begin{eqnarray}
 	\sqrt{n}(h_w(\widehat{\theta}_{W})-h_w(\theta)) &\overset{d}{\rightarrow}&
 h_w'(\theta)	K_W  \left[\delta \left[\begin{array}{l} \psi_0\\ 
\psi_1 \end{array}\right]  \otimes \iota_{T-2} + N\left(0, \mathcal{V}_W\right) \right].
	\end{eqnarray}

	
	Consider centering expression (31) to $h(\theta_n)$ where $\alpha_2 = \frac{\delta}{\sqrt{n}}$. Using Taylor expansion,
	
	\[
h(\theta_n) = h(\theta) + \bigtriangledown_{\alpha_2} h(\bar{\theta})'\frac{\delta}{\sqrt{n}}.		
	\]	
	
	Hence, we have
	\begin{eqnarray}
 			\sqrt{n} (h(\widehat{\theta}_{LW})-h(\theta_n)) &\overset{d}{\rightarrow}&  			N\left(0,\,\, h'(\theta)K_{LW}\mathcal{V}_{LW}K_{LW}'h'(\theta) \right) -\bigtriangledown_{\alpha_2} h(\theta)'\delta 
	\end{eqnarray}
	
	We can use $\widehat{\delta} = \sqrt{n}\,\, \widehat{\alpha}_{2, LW}$ which is asymptotically unbiased estimator of $\delta$.	
\begin{eqnarray}
 			\widehat{\delta} &\overset{d}{\rightarrow}& \delta + (0, 1, 0)K_{LW} \mathcal{V}_{LW} K_{LW}'(0, 1, 0)'	  			
	\end{eqnarray}	
	
