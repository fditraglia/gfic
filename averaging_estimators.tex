%!TEX root = main.tex
\section{Averaging and Post-Selection Inference}
\label{sec:avg}
While we are primarily concerned in this paper with the mean-squared error performance of our proposed selection techniques, it is important to have tools for carrying out valid inference post-selection.
To this end, we now show how to extend the results from Section 4 of \cite{DiTraglia2012} to the more general setting considered in this paper, one that allows for simultaneous model and moment selection.\footnote{Because the conceptual issues are largely the same as in the case where one considers only moment selection, we direct the reader to \cite{DiTraglia2012} for more discussion.}
Consider an estimator of the form 
	$$\widehat{\mu} = \sum_{(b,c) \in \mathcal{BC}} \widehat{\omega}(b,c) \widehat{\mu}(b,c)$$
  where $\widehat{\mu}$ denotes the target parameter under the moment conditions and parameter restrictions indexed by $(b,c)$, $\mathcal{BC}$ denotes the full set of candidate specifications, and $\widehat{\omega}(b,c)$ denotes a collection of data-dependent weights satisfying the following assumption.
\begin{assump}[Data-Dependent Weights] Let $\widehat{\omega}(b,c)$ be a function of the data $Z_{n1}, \hdots, Z_{nn}$ and $(b,c)$ satisfying
	\begin{enumerate}[(a)] 
		\item $\sum_{(b,c) \in \mathcal{BC}} \widehat{\omega}(b,c) = 1$
		\item $\widehat{\omega}(b,c) \overset{d}{\rightarrow} \psi(\mathscr{N}, \delta, \tau|b,c)$ jointly for all $(b,c) \in \mathcal{BC}$ where $\psi$ is a function of the normal random vector $\mathscr{N}$, the bias parameters $\delta$ and $\tau$, and consistently estimable quantities only.
	\end{enumerate}
\label{assump:weight}
\end{assump}

Assumption \ref{assump:weight} is quite weak, covering a broad range of examples, including genuine averaging estimators, post-GFIC estimators, and pre-test estimators based on the J-statistic.
Under this assumption, we can characterize the limit distribution of $\widehat{\mu}$ as follows.

\begin{cor}[Limit Distribution of Averaging Estimators]
Let $\widehat{\omega}(b,c)$ be a set of weights satisfying Assumption \ref{assump:weight}. Then, under the hypotheses of Theorem \ref{thm:asymp},  
	$$\sqrt{n}\left(\widehat{\mu} - \mu_n\right) \overset{d}{\rightarrow} \Lambda(\tau,\delta)$$
where
	\begin{equation}
		\Lambda(\tau,\delta) = -\nabla_\beta\varphi_0' \sum_{(b,c) \in \mathcal{BC}} \psi(\mathscr{N},\delta, \tau|b,c) \left\{\Xi_b' K(b,c) \Xi_c \mathscr{N} + M(b,c)  \left[\begin{array}{c}\delta \\ \tau \end{array} \right]\right\}
	\end{equation}
\end{cor}
Note that the limit distribution from the preceding corollary is highly non-normal: it is a \emph{randomly} weighted average of a normal random vector, $\mathscr{N}$.
To tabulate this distribution for the purposes of inference, we will in general need to resort to simulation.
If $\tau$ and $\delta$ were known, the story would end here.
We could simply substitute consistent estimators of $K$ and $M$, and then repeatedly draw $\mathscr{N} \sim N(0, \widehat{\Omega})$, where $\widehat{\Omega}$ is a consistent estimator of $\Omega$, and thus tabulate the distribution of $\Lambda$ to arbitrary precision. 
Unfortunately, no consistent estimators of $\tau$ or $\delta$ exist: all we have at our disposal are asymptotically unbiased estimators.
Simply plugging in these estimators $\widehat{\tau}$ and $\widehat{\delta}$ and proceeding with the simulation is not guaranteed to lead to valid confidence intervals.\footnote{Although it does not work in general, in particular examples this plug-in procedure may perform well. For more discussion of this point, see Section 4.4 of \cite{DiTraglia2012}.}
In contrast, the following two-step procedure, is guaranteed to yield confidence intervals with asymptotic coverage probability \emph{no less than} $1-(\alpha_1 + \alpha_2)$.

\begin{alg}[Simulation-based Confidence Interval for $\widehat{\mu}$]
\label{alg:conf}
\mbox{}
\begin{enumerate}
	\item Construct $ R(\alpha_1)$, a $(1-\alpha_1)\times 100\%$ joint confidence region for $(\delta,\tau)$ 
	\item For each $(\delta,\tau)\in R(\alpha_1)$:
		\begin{enumerate}[(i)]
			\item For each $j = 1, 2, \hdots, B$, generate $\mathscr{N}_j \sim N(0, \widehat{\Omega})$
			\item For each for $j = 1, 2, \hdots, J$ set $$\Lambda_j(\tau,\delta)= -\nabla_\beta\widehat{\varphi}_0' \sum_{(b,c) \in \mathcal{BC}} \widehat{\psi}(\mathscr{N}_j,\delta, \tau|b,c) \left\{\Xi_b' \widehat{K}(b,c) \Xi_c \mathscr{N}_j + \widehat{M}(b,c)  \left[\begin{array}{c}\delta \\ \tau \end{array} \right]\right\}$$
			\item Using $\{\Lambda_j(\delta, \tau)\}_{j=1}^J$, calculate $\widehat{a}(\delta,\tau)$, $\widehat{b}(\delta, \tau)$ such that
		$$P\left\{ \widehat{a}(\delta,\tau) \leq\Lambda(\delta,\tau)\leq \widehat{b}(\delta,\tau) \right\} = 1 - \alpha_2$$
		\end{enumerate}
	\item Define
			\begin{eqnarray*}
				\widehat{a}_{min}(\widehat{\delta}, \widehat{\tau})&=& \min_{(\delta,\tau) \in R(\alpha_1)} \widehat{a}(\delta,\tau)\\
				\widehat{b}_{max}(\widehat{\delta}, \widehat{\tau})&=& \max_{(\delta,\tau) \in R(\alpha_1)}\widehat{b}(\delta,\tau)
			\end{eqnarray*}
	\item The confidence interval for is $\mu$ given by
				$$\mbox{CI}_{sim}=\left[ \widehat{\mu} - \frac{\widehat{b}_{max}(\widehat{\delta}, \widehat{\tau})}{\sqrt{n}}, \;\;\; \widehat{\mu} - \frac{\widehat{a}_{min}(\widehat{\delta}, \widehat{\tau})}{\sqrt{n}} \right]$$
\end{enumerate}
\end{alg}

\begin{thm}[Simulation-based Confidence Interval for $\widehat{\mu}$]
\label{pro:sim}
Let $\nabla_{\beta}\widehat{\varphi}_0$, $\widehat{\psi}(\cdot|b,c)$, $\widehat{K}(b,c)$ and $\widehat{M}(b,c)$ be consistent estimators of $\nabla_\beta \varphi$, $\psi(\cdot|b,c)$, $K(b,c)$ and $M(b,c)$ and let $R(\alpha_1)$ be a $(1-\alpha_1)\times 100\%$ joint confidence region for $(\delta,\tau)$ constructed from Theorem \ref{thm:jointbias}.
Then the interval $CI_{sim}$ defined in Algorithm \ref{alg:conf} has asymptotic coverage probability no less than $1-\left( \alpha_1 + \alpha_2 \right)$ as $J,n\rightarrow \infty$. 
\end{thm}
