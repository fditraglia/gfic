\documentclass[11pt]{article}

\textwidth 16.5cm
\evensidemargin 0mm
\oddsidemargin0mm
\topmargin-18mm
\textheight 23.0cm

%----------------------------------------------------------------------------------------------------------
% PACKAGES
%----------------------------------------------------------------------------------------------------------
%\usepackage{epsfig}
%\usepackage{draftwatermark}
\usepackage{everypage}
\usepackage{lscape}
\usepackage{amsfonts}
\usepackage{amssymb}
\usepackage{amsmath}
\usepackage{setspace}
\usepackage[stable]{footmisc}
\doublespacing
%\usepackage[bottom]{footmisc}
\usepackage{float}
\usepackage{longtable}
\usepackage[center,bf]{caption}
\usepackage[debug,citecolor=blue]{hyperref}
\usepackage{theorem}
\usepackage[toc,page,title,titletoc,header]{appendix}
\usepackage{tikz}                                             % you only need this when using TikZ graphics
%\usepackage{pdfsync}
\usepackage{graphicx}
\usepackage{float}
\usepackage[final]{pdfpages}
%\usepackage{epsfig}
%\usepackage{3parttable}
%\usepackage{dcolumn}
\DeclareGraphicsExtensions{.png, .pdf, .jpg, .mps, .ps, .pdf, .eps}
\usepackage{pdfpages}
\usepackage{color}
\usepackage{xcolor}
\usepackage{rotating}
\usepackage{multirow}
\usepackage{placeins}

\usepackage{bbm}

\graphicspath{{./figuras/}}

%----------------------------------------------------------------------------------------------------------
% BIBLIOGRAPHY
%----------------------------------------------------------------------------------------------------------
\usepackage{natbib}
%{\bibliographystyle{econometrica}
\renewcommand{\cite}{\citet}

%--------
%%WATER MARK
%\SetWatermarkFontSize{1cm}
%\SetWatermarkScale{2}
%\SetWatermarkLightness{.9}
%\SetWatermarkText{Preliminary Draft}

%--------

%%%%%%%%% NEW COMMAND MACRO %%%%%%%%%%%%%%%%%%%%%%%%%%%%%%

\newcommand{\re}[1]{\smallskip\textsf{\textbf{Respuesta}} \begin{sf}\\
#1 \end{sf} \bigskip}
\newcommand{\pr}[2]{\frac{\partial #1}{\partial #2}}

\newcommand{\notorth}{\ensuremath{\perp\!\!\!\!\!\!\diagup\!\!\!\!\!\!\perp}}%
\newcommand{\orth}{\ensuremath{\perp\!\!\!\perp}}%

%%%%%%%%%%%%%%%%%%%%%%%%%%%%%%%%%%%%%%%%%%%%%%%%%%%%%%%%%%

%\input{ee.sty}

\setcounter{MaxMatrixCols}{10}

\newtheorem{example}{Example}[section]
\newtheorem{assume}{Assumption}
\newtheorem{proposition}{Proposition}[section]
\newtheorem{lemma}{Lemma}[section]
\newtheorem{corollary}{Corollary}[section]
\newtheorem{theorem}{Theorem}[section]
\newtheorem{definition}{Definition}[section]
\setcounter{section}{0}
\makeindex

\newcommand{\notindep}{\ensuremath{\perp\!\!\!\!\!\!\diagup\!\!\!\!\!\!\perp}}
\newcommand{\indep}{\ensuremath{\perp\!\!\!\perp}}
\newcommand{\Perp}{\perp \! \! \! \perp}

\newcommand{\prob}{\mathbb{P}}
\newcommand{\expec}{\mathbb{E}}


\def\inprobHIGH{\,{\buildrel p \over \longrightarrow}\,} 
\def\inprob{\,{\inprobHIGH}\,} 

\def\indistHIGH{\,{\buildrel d \over \longrightarrow}\,} 
\def\indist{\,{\indistHIGH}\,} 

%----------------------------------------------------------------------------------------------------------
% HYPER-REFERENCES CONFIGURATION - PDF
%----------------------------------------------------------------------------------------------------------

\hypersetup{
  colorlinks=true,linkcolor=blue,citecolor=black, filecolor=magenta,
citebordercolor=yellow,
  linkbordercolor = magenta,
  pdftitle={SGL},
  pdfauthor={Rojas},
  pdfcreator={\LaTeX\ with package \flqq hyperref\frqq using Textmate},
  pdfsubject={achievement},
  pdfkeywords={achievement}
}

\usepackage{psfrag}
\usepackage{afterpage}
\usepackage{subfigure}
\usepackage{setspace}
\usepackage{fancyhdr}
\usepackage{mathrsfs}
\newcommand{\norm}[1]{\left\lVert#1\right\rVert}

\setlength\parindent{0pt}

\begin{document}
%\title{\LARGE Problem Set 7}
%\author{Minsu Chang}
%\date{\small{\today}}


 %\maketitle \thispagestyle{empty}

\doublespacing

%\setcounter{page}{0}


\section*{Typos}

I put how we should change after the arrow $\rightarrow$. 

\begin{itemize}
\item p.13. Corollary 4.1. line 2. $\sqrt{n} (\widehat{\mu} - \mu_n) \rightarrow_d \Lambda(\tau, \delta) \quad \quad \rightarrow\quad \quad  \textcolor{red}{\Lambda(\delta, \tau)}$
\item Algorithm 4.1. (Approximating Quantiles of $\Lambda(\tau, \delta)) \qquad \qquad \rightarrow \qquad \qquad \textcolor{red}{\Lambda(\delta, \tau)}$
\item p.13. last paragraph line 3: $CI_1$ it does not necessarily yield... $\rightarrow$ $CI_1$ does not necessarily.. 
\end{itemize}

\section*{Proof of Theorem 4.1.}
By Theorem 3.1. and Corollary 4.1, 
\[
P\{ \mu_0 \in CI_{sim}\} \rightarrow P\{ a_{min} \leq \Lambda(\delta, \tau) \leq b_{max}\}
\]

where $a(\delta^{*}, \tau^{*}), b(\delta^{*}, \tau^{*})$ define a collection of $(1-\alpha_2) \times $ 100\% intervals indexed by $(\delta^{*}, \tau^{*})$, each of which is constructed under the assumption that $\delta=\delta^{*}, \tau=\tau^{*}$
\[
P\{a(\delta^{*},\tau^{*}) \leq \Lambda(\delta^{*}, \tau^{*})\leq b(\delta^{*}, \tau^{*})\} = 1-\alpha_2
\]

and we define the shorthand $a_{min}, b_{max}$ as follows
\begin{align*}
a_{min}\left( \begin{bmatrix}
\delta & \tau
\end{bmatrix}' + \Psi \mathscr{N} \right) &= \min\left \{ a(\delta^{*}, \tau^{*}): (\delta^{*}, \tau^{*}) \in \mathscr{R} \left( \begin{bmatrix}
\delta & \tau
\end{bmatrix}' + \Psi \mathscr{N}, \,\, \alpha_1 \right)  \right \}\\
b_{max}\left( \begin{bmatrix}
\delta & \tau
\end{bmatrix}' + \Psi \mathscr{N} \right) &= \max\left \{ b(\delta^{*}, \tau^{*}): (\delta^{*}, \tau^{*}) \in \mathscr{R} \left( \begin{bmatrix}
\delta & \tau
\end{bmatrix}' + \Psi \mathscr{N}, \,\, \alpha_1 \right)  \right \}\\
\mathscr{R}\left( \begin{bmatrix}
\delta & \tau
\end{bmatrix}' + \Psi \mathscr{N}, \,\, \alpha_1 \right ) &= \left \{ (\delta^{*}, \tau^{*}): \Delta \left( \begin{bmatrix}
\delta & \tau
\end{bmatrix}', \begin{bmatrix}
\delta^{*} & \tau^{*}
\end{bmatrix}' \right) \leq \chi_{p+q}^2(\alpha_1) \right \}\\
\Delta \left( \begin{bmatrix}
\delta & \tau
\end{bmatrix}', \begin{bmatrix}
\delta^{*} & \tau^{*}
\end{bmatrix}' \right)& = \left( \begin{bmatrix}
\delta & \tau
\end{bmatrix}' + \Psi \mathscr{N} - \begin{bmatrix}
\delta^{*} & \tau^{*}
\end{bmatrix}' \right)' (\Psi\Omega\Psi')^{-1}\left( \begin{bmatrix}
\delta & \tau
\end{bmatrix}' + \Psi \mathscr{N} - \begin{bmatrix}
\delta^{*} & \tau^{*}
\end{bmatrix}' \right)
\end{align*}

Now, let $A = \left \{\Delta \left( \begin{bmatrix}
\delta & \tau
\end{bmatrix}', \begin{bmatrix}
\delta^{*} & \tau^{*}
\end{bmatrix}' \right) \leq \chi_{p+q}^2(\alpha_1)  \right \}$ where $\chi_{p+q}^2(\alpha_1)$ is the $1-\alpha_1$ quantile of a $\chi_{p+q}^2$ random variable. This is the event that the limiting version of the confidence region for $(\delta, \tau)$ contains the true bias parameter. Since $\Delta \left( \begin{bmatrix}
\delta & \tau
\end{bmatrix}', \begin{bmatrix}
\delta & \tau
\end{bmatrix}' \right) \sim \chi_{p+q}^2$, $P(A) = 1-\alpha_1$. For every $(\delta^{*}, \tau^{*}) \in  \mathscr{R} \left( \begin{bmatrix}
\delta & \tau
\end{bmatrix}' + \Psi \mathscr{N}, \,\, \alpha_1 \right)$ we have
\[
P\left[\left\{ a(\delta^{*}, \tau^{*}) \leq \Lambda(\delta^{*}, \tau^{*}) \leq b(\delta^{*}, \tau^{*})  \right\} \cap A \right] + P\left[\left\{ a(\delta^{*}, \tau^{*}) \leq \Lambda(\delta^{*}, \tau^{*}) \leq b(\delta^{*}, \tau^{*})  \right\} \cap A^C \right] = 1-\alpha_2
\]

by decomposing $P\left\{ a(\delta^{*}, \tau^{*}) \leq \Lambda(\delta^{*}, \tau^{*}) \leq b(\delta^{*}, \tau^{*})  \right\}$ into the sum of mutually exclusive events. But since 
\[
P\left[\left\{ a(\delta^{*}, \tau^{*}) \leq \Lambda(\delta^{*}, \tau^{*}) \leq b(\delta^{*}, \tau^{*})  \right\} \cap A^C \right] \leq P(A^C) = \alpha_1
\]
we see that
\[
P\left[\left\{ a(\delta^{*}, \tau^{*}) \leq \Lambda(\delta^{*}, \tau^{*}) \leq b(\delta^{*}, \tau^{*})  \right\} \cap A \right] \geq 1- \alpha_1 - \alpha_2
\]
for every $ (\delta^{*}, \tau^{*}) \in  \mathscr{R} \left( \begin{bmatrix}
\delta & \tau
\end{bmatrix}' + \Psi \mathscr{N}, \,\, \alpha_1 \right)$. Now, by definition, if $A$ occurs then the true bias parameter is contained in $ \mathscr{R} \left( \begin{bmatrix}
\delta & \tau
\end{bmatrix}' + \Psi \mathscr{N}, \,\, \alpha_1 \right)$ and hence 
\[
P\left[\left\{ a(\delta, \tau) \leq \Lambda(\delta, \tau) \leq b(\delta, \tau)  \right\} \cap A \right] \geq 1- \alpha_1 - \alpha_2.
\]
But when $ (\delta^{*}, \tau^{*}) \in  \mathscr{R} \left( \begin{bmatrix}
\delta & \tau
\end{bmatrix}' + \Psi \mathscr{N}, \,\, \alpha_1 \right)$, $a_{min} \leq a(\delta, \tau)$ and $b(\delta, \tau) \leq b_{max}$. It follows that 
\[
\left \{ a(\delta, \tau) \leq \Lambda (\delta, \tau) \leq b(\delta, \tau) \right \} \cap A \subseteq \left\{ a_{min} \leq \Lambda (\delta, \tau) \leq b_{max}\right \}
\]
and therefore
\begin{align*}
1-\alpha_1 - \alpha_2 &\leq P\left[\left \{ a(\delta, \tau) \leq \Lambda (\delta, \tau) \leq b(\delta, \tau) \right \} \cap A  \right]\\
&\leq P\left[ a_{min} \leq \Lambda(\delta, \tau) \leq b_{max} \right]
\end{align*}
as asserted.
\end{document}