%!TEX root = main.tex
\section{Empirical Example: The Demand for Cigarettes}
\label{sec:cigarettes}
We now consider an empirical example illustrating the GFIC in the dynamic panel setting introduced in Section \ref{sec:Dpanel}.
Our exercise is based on \cite{BaltagiEtAl2000} who study the demand for cigarettes using panel data for 46 U.S.\ states between 1963 and 1992. 
Their model is
\[
  \ln C_{it} =  \gamma \ln C_{i,t-1} + \theta \ln P_{it} + \beta_1 \ln Y_{it} + \beta_2 \ln Pn_{it} + \eta_i + \lambda_t + v_{it}
\] 
where $C_{it}$ is the number of packs of cigarettes sold per person aged 16 and above, $P_{it}$ is the real average retail price of a pack of cigarettes, $Y_{it}$ is per capita disposable income, $Pn_{it}$ is the minimum average price of a pack of cigarettes in any state that neighbors state $i$, $\eta_i$ is a state fixed effect, and $\lambda_t$ is a time fixed effect.
The lagged dependent variable in this model is meant to capture habit-persistence in cigarette consumption but it is the price elasticity not the habit-persistence \emph{per se} that is of primary interest. 

\cite{BaltagiEtAl2000} consider an exhaustive list of possible estimators for two target parameters, the short-run price elasticity $\theta$ and the long-run price elasticity $\theta/(1 - \gamma)$, and explore how the resulting estimates vary.
Here we consider selecting between four alternative estimators of the short-run price elasticity $\theta$, as in the second simulation experiment from Section \ref{sec:Dpanel_sim}.
Each specification is estimated by TSLS in first differences, using the expressions from section \ref{sec:Dpanel}.\footnote{Our baseline specification is similar to the estimator that \cite{BaltagiEtAl2000} refer to as FD2SLS. There are two differences however. First, whereas they use lags of the exogenous controls $\ln Y_{it}$ and $\ln Pn_{it}$, our instrument set follows \cite{AndersonHsiao}. 
Second, whereas \cite{BaltagiEtAl2000} appear to have inadvertantly omitted the time dummies from their FD2SLS specification, we include them.}
For simplicity, we assume that the controls $\ln Y_{it}$ and $\ln Pn_{it}$ are exogenous with respect to $v_{it}$.
We focus on two questions.
First: what exogeneity assumption should we impose on $\ln P_{it}$?
Second: should we allow for habit persistence by estimating $\gamma$?

Our baseline specification, $\text{LP}$, estimates both $\gamma$ and $\theta$ and assumes only that $\ln P_{it}$ is predetermined with respect to $v_{it}$. 
This estimator uses the instrument set $\mathbf{z}_{it}(\ell, \text{P})$ with $\ell = 1$ from Equation \ref{eq:Zdpanel}.
For the purposes of this exercise we assume that $\text{LP}$ is correctly specified.
The specification $\text{LS}$ also estimates $\gamma$, but uses the expanded instrument set $\mathbf{z}_{it}(\ell, \text{S})$ with $\ell=1$ from equation \ref{eq:Zdpanel}.
The additional instruments used in $\text{LS}$ are only valid if $\ln P_{it}$ is strictly exogenous with respect to $v_{it}$.
Like $\text{LP}$ and $\text{LS}$, the specifications $\text{P}$ and $\text{S}$ differ in whether or not they impose that $\ln P_{it}$ is predetermined or strictly exogenous.
In contrast, however, they set $\gamma = 0$ and estimate a model with no habit persistence ($\ell =0$).
This increases the number of time periods available for estimation.


\begin{table}[htbp]
  \centering
    \begin{subtable}[h]{0.45\textwidth}
        \centering
     \caption{1975--1980 ($T=6$)}
     \label{tab:cigarettesShort}
     \begin{tabular}{lrrrr}\hline\hline 
         & \multicolumn{1}{c}{$\text{LP}$} & \multicolumn{1}{c}{$\text{LS}$} 
          & \multicolumn{1}{c}{$\text{P}$} & \multicolumn{1}{c}{$\text{S}$}\\
        \hline
        $\widehat{\theta}$ & -0.68 & -0.32 &  -0.28 &  -0.37\\
        Var.\ & 0.16 &  0.02 & 0.07 & 0.01\\ 
        Bias$^2$ & \multicolumn{1}{c}{---} & -4.20 & 0.01 & -3.56 \\
        GFIC  & 0.16 & -4.18 & 0.08  & -3.54  \\
        GFIC+  & 0.16 &  0.02 & 0.08  & 0.01    \\
        \hline
      \end{tabular}
    \end{subtable}
    ~
    \begin{subtable}[h]{0.45\textwidth}
      \centering
     \caption{1975--1985 ($T=11$)}
     \label{tab:cigarettesLong}
     \begin{tabular}{lrrrr}\hline\hline 
         & \multicolumn{1}{c}{$\text{LP}$} & \multicolumn{1}{c}{$\text{LS}$} 
          & \multicolumn{1}{c}{$\text{P}$} & \multicolumn{1}{c}{$\text{S}$}\\
        \hline
        $\widehat{\theta}$ & -0.30 & -0.26 &  -0.38 &  -0.28\\
        Var.\ & 0.06 & 0.01 & 0.05 & 0.01\\ 
        Bias$^2$ & \multicolumn{1}{c}{---} & 2.21 & 0.03 & 1.29\\
        GFIC  &0.06 & 2.22 & 0.08 &  1.30\\
        GFIC+  & 0.06 & 2.22 & 0.08  & 1.30  \\
           \hline
      \end{tabular}
    \end{subtable}
    \caption{Estimates and GFIC values for the price elasticity of demand for cigarettes example from Section \ref{sec:cigarettes} under four alternative specifications. Panel (a) presents results using data from 1975--1980 while Panel (b) presents results using data from 1975--1985. 
GFIC+ gives an alternative version of the GFIC in which a negative squared bias estimate is set equal to zero.}
    \label{tab:cigarettes}
\end{table}

Estimates and GFIC results for all specifications appear in Table \ref{tab:cigarettes}.
In Panel \ref{tab:cigarettesShort} we use data from 1975--1980 only ($T=6$).
After first-differencing, this leaves 4 time periods for estimation in specifications that include a lag ($\text{LP}$ and $\text{LS}$) versus 5 for those that do not ($\text{P}$ and $\text{S}$).
In panel \ref{tab:cigarettesLong} we use data from 1975--1985 ($T=11$).
After first-differencing this leaves 9 time periods for estimation without a lag versus 10 for estimation with a lag.
We choose to artificially shorten the time dimension of the panel from \cite{BaltagiEtAl2000} to illustrate a key feature of the GFIC, namely that it takes into account the number of available time periods when selecting over parameter restrictions and moment conditions.\footnote{Appendix \ref{sec:append_empirical} presents additional results, including the full-sample estimates, and some further discussion.} 
Each column in Table \ref{tab:cigarettes} refers to a particular specification: $\text{LP}$, $\text{LS}$, $\text{P}$ or $\text{S}$.
The first row of each panel gives the associated estimate of the target parameter $\theta$ while the second gives the estimated asymptotic variance of $\sqrt{n}(\widehat{\theta} - \theta_0)$, one of the two ingredients of the GFIC.\footnote{We estimate the asymptotic variance matrix as in \cite{BaltagiEtAl2000}.}
Unsurprisingly the asymptotic variance decreases with the number of time periods available for estimation: for a given sample period $\text{LP}$ and $\text{LS}$ have a higher asymptotic variance than $\text{P}$ and $\text{S}$, and all the estimates based on the 1975--1985 sample are more precise than their counterparts for the 1975-1980 sample.
Moreover, for a given sample period, the estimators with a large instrument set show a lower asymptotic variance: $\text{LS}$ is more precisely estimated than $\text{LP}$ and $\text{S}$ is more precisely estimated than $\text{P}$.

The third row of each panel gives our estimate of the squared asymptotic bias of the various estimators of $\theta$.
The ``---'' entry for the $\text{LP}$ estimator indicates that the GFIC is constructed under the assumption that this specification has no asymptotic bias.
Note that the squared bias estimator is \emph{negative} for $\text{LS}$ and $\text{S}$ in the 1975--1980 sample.
This occurs when the second term in the estimate of the bias matrix $\widehat{B}$ from Corollary \ref{cor:biasestimators}, namely $\widehat{\Psi}\widehat{\Omega}\widehat{\Psi}'$, is larger than the first.
Accordingly, we consider two alternative ways of constructing the GFIC from the asymptotic variance and squared bias estimates.
The first, labelled ``GFIC,'' simply adds squared bias and variance.
The second, labelled ``GFIC+,'' first \emph{truncates} a negative squared bias estimate to zero, and then adds the result to the variance estimate.
For the 1975--1980 sample period we find no evidence of appreciable bias in any of the three ``suspect'' specifications: $\text{LS}$, $\text{P}$, and $\text{LP}$.
In contrast, each of these has a substantially small asymptotic variance than $\text{LP}$, so we would select either $\text{LS}$ or $\text{S}$ depending on whether we prefer to use GFIC or GFIC+.\footnote{When GFIC and GFIC+ disagree, we prefer GFIC+ for the same reason that the positive-part Stein estimator is preferred to the ``plain-vanilla'' Stein estimator.}
In the 1975--1985 sample, the situation changes drastically.
Over this longer time period, the relative advantage of $\text{LS}$, $\text{P}$ and $\text{S}$ over $\text{LP}$ in asymptotic variance decreases substantially and we find evidence of substantial bias in both the $\text{LS}$ and $\text{S}$ specifications.
It appears that over these additional time periods, the assumption that $\ln P_{it}$ is strictly exogenous fails.
Interestingly the difference in GFIC values between $\text{LP}$ and $\text{P}$ in this longer sample is negligible.
As $\text{P}$ has a lower GFIC value than $\text{LP}$ in the 1975--1980 sample, it appears that accounting for habit persistence is relatively unimportant in estimating the short-run price elasticity of cigarette demand.

